% Copyright 2013 Nicolai Hähnle <nhaehnle@gmail.com>
%
% This work is licensed under the Creative Commons Attribution-ShareAlike 3.0
% Unported License, see http://creativecommons.org/licenses/by-sa/3.0/
%
% Among other things, this means that yes, you may take e.g. illustrations from
% the book and use them in your own work. However, (a) you must give proper
% attribution by naming me as its original author and (b) you must make your
% derivative work available under the same or similar license terms.
%
% See the Creative Commons website for the exact licensing terms.

\documentclass[a4paper,10pt]{scrbook}
\usepackage[utf8]{inputenc}

\usepackage{amsmath}
\usepackage{amssymb}
\usepackage{amsthm}
\usepackage{clrscode}
\usepackage{multirow}
\usepackage{enumerate}
\usepackage{tikz}

\usetikzlibrary{calc}
\usetikzlibrary{intersections}
\usetikzlibrary{arrows}
\usetikzlibrary{decorations.pathmorphing}

\newcommand{\E}{\mathbb{E}}
\newcommand{\N}{\mathbb{N}}
\newcommand{\Q}{\mathbb{Q}}
\newcommand{\R}{\mathbb{R}}
\newcommand{\Z}{\mathbb{Z}}
\newcommand{\C}{\mathbb{C}}
\newcommand{\cA}{\mathcal{A}}
\newcommand{\cD}{\mathcal{D}}
\newcommand{\cI}{\mathcal{I}}
\newcommand{\cP}{\mathcal{P}}
\newcommand{\cV}{\mathcal{V}}
\providecommand{\one}{\mathbf{1}}
\DeclareMathOperator{\vol}{vol}
\DeclareMathOperator{\cone}{cone}
\DeclareMathOperator{\tcone}{tcone}
\DeclareMathOperator{\conv}{conv}
\DeclareMathOperator{\diam}{diam}
\DeclareMathOperator{\poly}{poly}
\DeclareMathOperator{\Ker}{Ker}
\DeclareMathOperator{\Var}{Var}
\DeclareMathOperator{\Id}{id}
\DeclareMathOperator{\tr}{tr}
\DeclareMathOperator{\relint}{relint}
\DeclareMathOperator{\sgn}{sgn}

\theoremstyle{plain}
\newtheorem{theorem}{Theorem}[chapter]
\newtheorem{lemma}[theorem]{Lemma}
\newtheorem{proposition}[theorem]{Proposition}
\newtheorem{claim}[theorem]{Claim}
\newtheorem{corollary}[theorem]{Corollary}
\newtheorem{fact}[theorem]{Fact}
\newtheorem{conjecture}{Conjecture}

\theoremstyle{definition}
\newtheorem{definition}[theorem]{Definition}
\newtheorem{notation}[theorem]{Notation}
\newtheorem{example}[theorem]{Example}
\newtheorem{remark}[theorem]{Remark}
\newtheorem{problem}[theorem]{Problem}

\usepackage[colorlinks]{hyperref}

%opening
\title{Lattices and Convex Bodies}
\author{Nicolai Hähnle}

%\includeonly{chapter07-generating-functions}

\begin{document}

\maketitle

\tableofcontents

\chapter*{Preface}

These notes are being prepared as part of a lecture ``Lattices and Convex Bodies''
given by Nicolai Hähnle at the University of Bonn in Fall 2013.

The most recent version can be found at \url{https://github.com/nhaehnle/lattices}
in the form of a Git repository of the \LaTeX{} sources.

This work is licensed under the Creative Commons Attribution-ShareAlike 3.0
Unported License (CC BY-SA 3.0). See \url{http://creativecommons.org/licenses/by-sa/3.0/}
for detailed information about what this means.

Contributions of spotted errors, improved proofs, and so on are welcome,
and will be added with proper attribution.
See the Github site linked above for more information.


\section*{Contributions}

Typos and other corrections have been contributed by:
\begin{quote}
  Anna Hermann\\
  Rasmus Schroeder\\
\end{quote}


% Copyright 2013 Nicolai Hähnle <nhaehnle@gmail.com>
%
% This work is licensed under the Creative Commons Attribution-ShareAlike 3.0
% Unported License, see http://creativecommons.org/licenses/by-sa/3.0/
%
% Among other things, this means that yes, you may take e.g. illustrations from
% the book and use them in your own work. However, (a) you must give proper
% attribution by naming me as its original author and (b) you must make your
% derivative work available under the same or similar license terms.
%
% See the Creative Commons website for the exact licensing terms.

\chapter{Lattice Basics and Minkowski's theorem}

Let us begin with an old question:
\begin{problem}
  When can a natural number $n$ be expressed as the sum of two squares, that is,
  when can we write $n = x^2 + y^2$ where $x$ and $y$ are integers?
\end{problem}
Suppose that $n = ab$, where both $a$ and $b$ are sums of two squares of integers, that is
\begin{align*}
  a &= x^2 + y^2 \\
  b &= z^2 + w^2
\end{align*}
Then
\begin{align*}
  n &= ab \\
    &= (x^2 + y^2)(z^2 + w^2) \\
    &= x^2 z^2 + x^2 w^2 + y^2 z^2 + y^2 w^2 \\
    &= (xz + yw)^2 + (xw - yz)^2
\end{align*}
That is, every product of sums of two squares can be written as a sum of two squares.
This suggests we should focus on indivisible factors, i.e. prime numbers.

\begin{problem}
  \label{problem:prime-sum-of-squares}
  When can a prime $p$ be expressed as the sum of two squares, that is,
  when can we write $p = x^2 + y^2$ where $x$ and $y$ are integers?
\end{problem}
The sum of squares reminds us of Pythagoras' theorem:
\begin{center}
  \begin{tikzpicture}
    \draw (0,0) -- node[below] {$x$} (3,0) -- node[right] {$y$} (3,2)-- node[above] {$\sqrt{p}$} (0,0);
  \end{tikzpicture}
\end{center}
We can rephrase Problem~\ref{problem:prime-sum-of-squares}:
For which primes $p$ is there an integer point on the circle of radius $\sqrt{p}$ around the origin?

Geometry alone cannot answer this question. Let us look at some examples:
\[ \mathbf{2}, 3, \mathbf{5}, 7, 11, \mathbf{13}, \mathbf{17}, 19, 23, \mathbf{29}, 31, \mathbf{37}, \mathbf{41}, 43, 47, \mathbf{53}, 59, \mathbf{61}, \dots \]
Ignoring the special case of the rather odd prime $2$,
the primes that can be written as a sum of two squares appear to be exactly those that are congruent to $1$ modulo $4$.
Indeed, $x^2 \equiv 0$ or $1 \pmod{4}$ for all integers $x$, so $x^2 + y^2 \equiv 0, 1, 2 \pmod{4}$.
\begin{fact}
  If $p \equiv 3 \pmod{4}$, then $p$ cannot be written as the sum of two squares.
\end{fact}
Let us consider a different modular arithmetic angle.
Suppose we have $p = x^2 + y^2$, then certainly $x^2 + y^2 \equiv 0 \pmod{p}$.
Since $p$ is a prime, $\Z / (p)$ is a field and we can rearrange to get
\[ (xy^{-1})^2 \equiv -1 \pmod{p}. \]
That is, we have a square root of $-1$.
Algebra tells us exactly when the field $\Z/(p)$ contains such a square root:
$(\Z/(p))^\star$ is cyclic of order $p-1$, so there exists an element $q$ of order $4$ if and only if $4 | (p - 1)$,
which is another way of saying that there is a square root $q$ of $-1$ if and only if $p \equiv 1 \pmod{4}$.

In this case, given any integer $a$, we have
\[ a^2 + (aq)^2 \equiv 0 \pmod{p}, \]
which means $a^2 + (aq)^2$ is -- if not equal to $p$ -- at least a multiple of $p$.
Can we use this observation to express $p$ as the sum of squares of two integers?

We certainly want both integers to be in the range $\{ 1, \dots, p - 1 \}$.
Even if $a$ satisfies this condition, $qa$ is likely to fall outside this range.
Luckily, $a^2 + (bp + aq)^2$ is a multiple of $p$ as well, for any integer $b$.
Is there a choice of $a$ and $b$ such that $a^2 + (bp + aq)^2 = p$?
Or, to put it differently, is there a point
\[
  x =
  \underbrace{\begin{pmatrix}
    q & p \\
    1 & 0
  \end{pmatrix}}_{=: B}
  \begin{pmatrix}
    a \\ b
  \end{pmatrix},
  a, b \in \Z
\]
such that $\|x\|_2 = \sqrt{p}$?

We have returned to the \emph{geometric} formulation of our problem,
except that we reduced the set of candidate points to a \emph{proper subset} of $\Z^2$ with a very specific structure.
This chapter develops the tools that will allow us to exploit this structure and solve Problem~\ref{problem:prime-sum-of-squares}.

\section{Basic definitions}

\begin{definition}
  A \emph{lattice} $\Lambda$ is a discrete additive subgroup of $\R^d$.
  Its \emph{rank} or \emph{dimension} $\dim\Lambda$ is the dimension of the linear span of $\Lambda$.
\end{definition}

In this context, \emph{discrete} means:
for every $x \in \Lambda$, there is an $\varepsilon > 0$ such that
$\|y - x\|_2 \geq \varepsilon$ for all $y \in \Lambda \setminus \{ x \}$.

Due to the additive structure of $\Lambda$, the quantifiers can be exchanged in the previous statement.
Let $\varepsilon > 0$ be such that $\|y\|_2 \geq \varepsilon$ for all $y \in \Lambda \setminus \{ 0 \}$.
Now, is it possible that there are some $x' \neq y' \in \Lambda$ with $\|y' - x'\|_2 < \varepsilon$?
No, because $y' - x' \in \Lambda \setminus \{ 0 \}$.
From this, a simple volume packing argument shows:
\begin{lemma}
  \label{lemma:finitely-many-points-in-bounded-region}
  A bounded region of space contains only finitely many points of any lattice.
\end{lemma}
\begin{proof}
  Let $\Lambda$ be a lattice and $R > 0$.
  It is sufficient to show that the ball of radius $R$ around the origin,
  which we write as $B(0,R)$, contains only finitely many points of $\Lambda$.

  Let $\varepsilon > 0$ such that $\|y\|_2 > \varepsilon$ for all $y \in \Lambda \setminus \{ 0 \}$.
  Note that the balls $B(x,\varepsilon/2)$ and $B(y,\varepsilon/2)$ are disjoint for $x \neq y \in \Lambda$,
  see Figure~\ref{fig:finitely-many-points-in-bounded-region}.
  Therefore,
  \[
    \vol B(0,R + \varepsilon/2) \geq \vol \bigcup_{x \in B(0,R)\cap \Lambda} B(x,\varepsilon/2) = \#(B(0,R) \cap \Lambda) B(x,\varepsilon/2),
  \]
  from which it follows that $B(0,R) \cap \Lambda$ is finite.
\end{proof}
\begin{figure}
  \begin{center}
  \begin{tikzpicture}
    \foreach \a/\b in {0/0,1/0,0/1,1/1,0/2,1/2,1/-1,2/-1,1/-2,
    -1/0,0/-1,-1/-1,0/-2,-1/-2,-1/1,-2/1,-1/2
    }
      \draw[fill=black!10] ($\a*(1,0.1) + \b*(0.1,0.7)$) circle[radius=0.3cm];

    \draw[thick] (0,0) circle[radius=2cm];
    \draw[thick] (0,0) circle[radius=2.3cm];

    \clip (-3.1,-3.1) rectangle (3.1,3.1);
    \foreach \a in {-7,-6,...,7.1}
      \foreach \b in {-4,-3,...,4.1}
        \fill ($\a*(1,0.1) + \b*(0.1,0.7)$) circle[radius=2pt];

    \draw (0,0) -- node[below,near end] {$R$} (2,0);
  \end{tikzpicture}
  \end{center}
  \caption{A volume argument shows that a bounded region contains only finitely many lattice points.}
  \label{fig:finitely-many-points-in-bounded-region}
\end{figure}

\begin{corollary}
  Every lattice (except for $\Lambda = \{ 0 \}$) has a shortest non-zero vector.
\end{corollary}

Note that every lattice has at least two shortest vectors,
and we have already seen lattices like $\Z^d$ that have more shortest vectors.
We will see an upper bound on the number of shortest vectors in chapter~\ref{chapter:not-yet}.

\begin{notation}
  The length of a shortest (non-zero) vector in the lattice is usually denoted as $\lambda_1(\Lambda)$
  or just $\lambda_1$ when the lattice is clear from the context.
\end{notation}



\begin{example}
  \begin{enumerate}
    \item $\Z^d$ is a lattice.

    \item Given an invertible matrix $B \in \R^{d \times d}$, the set
      \[ \Lambda(B) := \{ B t ~:~ t \in \Z^d \} \]
      is a lattice.

    \item Given rational vectors $b_1, \ldots, b_m \in \Q^d$, the set
      \[ \Lambda(b_1, \ldots, b_m) := \{ \sum_{j=1}^m t_j b_j ~:~ t_j \in \Z \} \]
      is a lattice.

    \item The condition of rationality cannot simply be dropped.
      When $d = 1$, $b_1 = 1$, $b_2 = \alpha$, where $\alpha$ is any irrational number,
      then the set $\Lambda(b_1, b_2)$ is dense in $\R$ and therefore not a lattice.

      This follows from the equidistribution theorem proved by Weyl, Sierpinski, and others.
      It can be seen in the context of Diophantine approximation, which we will see a bit of later.
  \end{enumerate}
\end{example}

\begin{definition}
  Given a lattice $\Lambda$,
  a \emph{basis} of $\Lambda$ is a linearly independent set of vectors $b_1, \ldots, b_k$
  such that $\Lambda = \Lambda(b_1, \ldots, b_k)$.
\end{definition}

We often think of basis vectors as column vectors of a matrix $B = (b_1, \ldots, b_k)$,
and write $\Lambda(B)$ for the lattice generated by $B$.
Clearly, $\dim\Lambda = k$.
We will mostly restrict our attention to full-dimensional lattices, i.e. the case $k = d$.

Every lattice has a basis; we postpone the proof of this fact until section~\ref{sec:bases}.
Except for the trivial case $d \leq 1$, a lattice has infinitely many bases,
and the choice of basis matters a great deal for computational problems.
We will discuss some related issues in chapter~\ref{chapter:basis-reduction-LLL}.

For now, just let $B$ be an invertible matrix,
which we think of as a lattice basis of the full-dimensional lattice $\Lambda = \Lambda(B)$.

Let $B'$ be another basis of $\Lambda$.
By definition, the vectors in $B'$ can be expressed as integer linear combinations of the vectors in $B$
and vice versa. Hence there are matrices $U, V \in \Z^{d \times d}$ such that
\[ B' = BU, B = B'V. \]
Taken together, this implies $V = U^{-1}$.
Since $\det$ is a group homomorphism, it follows that $\det(U) = \det(U^{-1}) = \pm 1$.

This implies that the absolute value of $\det(B)$ is an invariant of the lattice
and justifies the following definition:
\begin{definition}
  The \emph{determinant} of a full-dimensional lattice $\Lambda$
  is defined via $\det(\Lambda) := |\det(B)|$,
  where $B$ is any basis of $\Lambda$.
\end{definition}

In Section~\ref{sec:determinant-general-lattices},
we will see that this definition can be extended to general, not full-dimensional lattices.
For now, we will work only with the simpler definition given here.

\begin{definition}
  A matrix $U \in \Z^{d \times d}$ with $\det(U) = \pm 1$ is called \emph{unimodular}.
\end{definition}

\begin{lemma}
  \label{lemma:basis-exchange-is-unimodular}
  Let $\Lambda \subset \R^d$ be a lattice of dimension $k$ and let $B \in \R^{d \times k}$ be a basis of $\Lambda$.
  Then for $U \in \R^{k \times k}$ we have that $BU$ is a basis of $\Lambda$ if and only if $U$ is unimodular.
\end{lemma}
\begin{proof}
  The implication from left to right follows from the discussion above.
  For the reverse implication, one first sees that $\Lambda(BU) \subseteq \Lambda$ because $U$ is integral.
  Then we note that $U^{-1}$ is also integral because $\det(U) = \pm 1$,
  which implies that $\Lambda = \Lambda(BU\cdot U^{-1} \subseteq \Lambda(BU)$.
  Thus we have $\Lambda = \Lambda(BU)$, which completes the proof.
\end{proof}




\begin{definition}
  Let $\Lambda(B)$ be a lattice.
  The \emph{fundamental parallelepiped of $B$} is
  \[ \cP_B := \{ x = \sum_{j=1}^d \lambda_j b_j ~:~ 0 \leq \lambda_j < 1 \forall j = 1 \ldots n \}. \]
  We omit the subscript when the basis is clear from the context.
\end{definition}
See Figure~\ref{fig:fundamental-parallelepiped} for an illustration.
Observe that $\cP_B$ is the image of the half-open unit cube
under the linear transformation given by $B$. Hence
\[ \vol(\cP_B) = |\det(B)| \]

\begin{figure}
  \begin{center}
  \begin{tikzpicture}
    \clip (-2.1,-1.1) rectangle (6.1,3.1);
    \foreach \a in {-7,-6,...,7.1}
      \foreach \b in {-4,-3,...,4.1}
        \fill ($\a*(1,0.1) + \b*(0.1,0.7)$) circle[radius=2pt];

    \draw (0,0) node[below right] {$0$};

    \draw[fill=black!10] (0,0) -- (2.1,0.9) -- (3.2,1.7) -- (1.1,0.8) -- cycle;

    \draw[thick,->] (0,0) -- (2.1,0.9) node[below] {$b_2$};
    \draw[thick,->] (0,0) -- (1.1,0.8) node[above left] {$b_1$};
  \end{tikzpicture}
  \end{center}
  \caption{The fundamental parallelepiped of a lattice.}
  \label{fig:fundamental-parallelepiped}
\end{figure}



\section{Minkowski's theorem}

The power of lattices comes from studying them in relation to sets in $\R^d$.
A \emph{convex body} is a full-dimensional compact convex set.
Following the natural group theoretic definitions, we use
\[ x \equiv y \pmod{\Lambda} \]
to mean $x - y \in \Lambda$.
In this case, we say that $x$ is congruent to $y$ modulo $\Lambda$.

\begin{theorem}[Blichfeldt's theorem]
  \label{thm:blichfeldt}
  Let $\Lambda \subset \R^d$ be a lattice and $k \in \N$.
  Let $M \subseteq \R^d$ be a measurable set with $\vol(M) > k \det(\Lambda)$.
  Then there exist $k+1$ pairwise different points $x_1, \ldots, x_{k+1} \in M$
  that are congruent modulo $\Lambda$.
\end{theorem}
\begin{figure}
  \begin{center}
  \begin{tikzpicture}
    \def\M{
      \draw[thick,fill=gray,fill opacity=0.2]
        (-0.3,-0.5) -- (2.2,1.4) -- (5.9,3.1) -- (3.0,1.4) -- (3.2,1.2) -- cycle;
      \draw[thick,fill=gray,fill opacity=0.2]
        (5.2,1.9) circle[x radius=0.8,y radius=0.2,rotate=30];
    }

    \begin{scope}[shift={(0,-1)}]
      \clip (-1.1,-1.6) rectangle (6.6,3.5);
      \foreach \a in {-7,-6,...,7.1}
        \foreach \b in {-4,-3,...,4.1}
          \fill ($\a*(2,0.2) + \b*(0.2,1.4)$) circle[radius=2pt];

      \foreach \a in {-8,-7,...,8.1}
        \draw ($\a*(4.2,1.8)$) +($-9*(2.2,1.6)$) -- +($9*(2.2,1.6)$);

      \foreach \b in {-8,-7,...,8.1}
        \draw ($\b*(2.2,1.6)$) +($-9*(4.2,1.8)$) -- +($9*(4.2,1.8)$);

      \M
    \end{scope}

    \begin{scope}[shift={(6.8,0.1)}]
      \draw (0,0) -- (4.2,1.8) -- (6.4,3.4) -- (2.2,1.6) -- cycle;
      \clip (0,0) -- (4.2,1.8) -- (6.4,3.4) -- (2.2,1.6) -- cycle;

      \M
    \end{scope}

    \begin{scope}[shift={(6.8,-0.9)}]
      \draw (0,0) -- (4.2,1.8) -- (6.4,3.4) -- (2.2,1.6) -- cycle;
      \clip (0,0) -- (4.2,1.8) -- (6.4,3.4) -- (2.2,1.6) -- cycle;

      \begin{scope}[shift={(2.2,1.6)}]
        \M
      \end{scope}
    \end{scope}

    \begin{scope}[shift={(6.8,-1.9)}]
      \draw (0,0) -- (4.2,1.8) -- (6.4,3.4) -- (2.2,1.6) -- cycle;
      \clip (0,0) -- (4.2,1.8) -- (6.4,3.4) -- (2.2,1.6) -- cycle;

      \begin{scope}[shift={(-2,-0.2)}]
        \M
      \end{scope}
    \end{scope}

    \begin{scope}[shift={(6.8,-3.4)}]
      \draw (0,0) -- (4.2,1.8) -- (6.4,3.4) -- (2.2,1.6) -- cycle;
      \clip (0,0) -- (4.2,1.8) -- (6.4,3.4) -- (2.2,1.6) -- cycle;

      \foreach \a in {-7,-6,...,7.1} {
        \foreach \b in {-4,-3,...,4.1} {
          \begin{scope}[shift={($\a*(2,0.2) + \b*(0.2,1.4)$)}]
            \M
          \end{scope}
        }
      }
    \end{scope}

    \foreach \d in {-1,0,1}
      \draw[very thick,->] ($(10.0,-0.7) + \d*(0.35,0.15)$) -- +(0.0,-0.55);
  \end{tikzpicture}
  \end{center}
  \caption{Proof of Blichfeldt's theorem.}
  \label{fig:blichfeldt-theorem}
\end{figure}
\begin{proof}[Proof sketch]
  This result is best thought of as a variant of the pigeon hole principle.
  The fundamental parallelepiped $\cP$ associated to any basis of $\Lambda$
  tiles the space.
  We cut $M$ along those tiles and translate all the pieces by a lattice vector
  back to the fundamental parallelepiped,
  see Figure~\ref{fig:blichfeldt-theorem}.
  Then there must be a point in $\cP$ which is covered by at least $k+1$ of those translated pieces.
  This intuition can be formalized using integrals over characteristic functions.
\end{proof}

\begin{theorem}[Minkowski's first theorem]
  \label{thm:minkowski-first}
  Let $\Lambda$ be a lattice
  and let $M$ be a convex measurable set that is symmetric with respect to the origin
  and has $\vol(M) > 2^d \det(\Lambda)$.
  Then $M$ contains a non-zero lattice vector.

  If we replace ``convex set'' by ``convex body'',
  then $\vol(M) \geq 2^d \det(\Lambda)$ is a sufficient condition for the same result.
\end{theorem}
\begin{proof}[Proof sketch]
  Applying Blichfeldt's theorem to $\frac{1}{2} M$
  yields two points whose difference is a non-zero lattice vector in $M$.
  If $M$ is a convex body,
  then $(1 + \varepsilon) M$ contains no additional lattice vector for sufficiently small $\varepsilon > 0$,
  and we can apply the previous argument.
\end{proof}


\section{Applying Minkowski's theorem}

Let us return to Problem~\ref{problem:prime-sum-of-squares} from the beginning of the chapter.
We concluded with the question: Is there a point
\[
  x =
  \underbrace{\begin{pmatrix}
    q & p \\
    1 & 0
  \end{pmatrix}}_{=: B}
  \begin{pmatrix}
    a \\ b
  \end{pmatrix},
  a, b \in \Z
\]
such that $\|x\|_2 = \sqrt{p}$?
We can now answer this question in the affirmative.

Recall that $q \in \Z$ was chosen such that $a^2 + (bp + aq)^2$ is a multiple of $p$,
and therefore $\|x\|_2^2$ is a multiple of $p$ for every $x \in \Lambda(B)$.
It is therefore sufficient to show that the disk around the origin with radius $\sqrt{2p}$
contains a non-zero lattice point in its interior.

Let $D$ be the disk around the origin with radius $\sqrt{2p}$, see Figure~\ref{fig:prime-sum-of-squares}.
We have $\vol(D) = 2\pi p$ and $\det \Lambda = |\det B| = p$.
Since $\vol(D) > 4 \det \Lambda$,
we can apply Minkowski's Theorem~\ref{thm:minkowski-first}
to the interior of $D$,
which completes the solution of Problem~\ref{problem:prime-sum-of-squares}:
\begin{theorem}
  Let $p$ be an odd prime.
  Then $p$ is the sum of two squares if and only if $p \equiv 1 \pmod{4}$.
\end{theorem}

\begin{figure}
  \begin{center}
  \begin{tikzpicture}
    % lattice coordinates scaled to 60%
    \clip (-4,-2.8) rectangle (5,2.8);
    \foreach \a in {-7,-6,...,7.1}
      \foreach \b in {-6,-5,...,6.1}
        \fill ($\a*(3,0) + \b*(1.8,0.6)$) circle[radius=2pt];

    \draw[thick,->] (0,0) -- (1.8,0.6) node[above] {$b_2$};
    \draw[thick,->] (0,0) -- (3,0) node[below] {$b_1$};

    \draw (0,0) node[below] {$0$};

    \draw[thick,fill=gray,fill opacity=0.2] (0,0) circle[radius=1.897];
  \end{tikzpicture}
  \end{center}
  \caption{Applying Minkowski's theorem to Problem~\ref{problem:prime-sum-of-squares} for $p = 5$ and $q = 3$.}
  \label{fig:prime-sum-of-squares}
\end{figure}


For another classic application of Minkowski's theorem,
consider the problem of Simultaneous Diophantine Approximation.
Given numbers $\alpha_1, \ldots, \alpha_n \in \R$,
we want to find a single $q \in \N$ and $p_1,\ldots, p_n \in \Z$
such that $|p_j - q\alpha_j|$ is small for all $j$.




\section{The existence of bases}
\label{sec:bases}

We can already see that \emph{short but non-zero} lattice vectors play an important role:
representing a prime number as a sum of two squares can be achieved by finding a shortest non-zero vector in a certain lattice.
Diophantine approximations can be found as a short non-zero vectors with respect to the $\ell_\infty$ norm.
Later, we will see the relationship between short vectors and the integer programming problem.
Lattice bases play an essential role in lattice algorithms, because they are used for the numerical representation of lattices.
In this section, we will show a result that was promised earlier.

\begin{lemma}
  \label{lemma:every-lattice-has-basis}
  Every lattice $\Lambda$ has a basis.
\end{lemma}
\begin{proof}
  The proof of this theorem proceeds by induction over $d = \dim \Lambda$.
  The cases $d = 0$ and $d = 1$ are easy to see.

  Let $b_1 \in \Lambda \setminus \{ 0 \}$ be a \emph{primitive} vector,
  that is, a lattice vector such that $\lambda b_1 \not\in \Lambda$ for $\lambda \in (0,1)$
  (for example, one might take a shortest vector of the lattice).
  Let
  \begin{align*}
    \pi_1 : \R^d & \to \langle b_1 \rangle^\bot \\
    x & \mapsto x - \frac{x^T b_1}{b_1^T b_1} b_1
  \end{align*}
  be the orthogonal projection onto the orthogonal complement of the line spanned by $b_1$.
  \begin{lemma}
    $\pi_1(\Lambda)$ is a lattice.
  \end{lemma}
  \begin{proof}
    It is clear that $\pi_1(\Lambda)$ is a subgroup of $\R^d$.

    Now, note that every $x' \in \pi_1(\Lambda)$ has a pre-image $x \in \Lambda$ that satisfies
    \[
      \frac{x^T b_1}{b_1^T b_1} \in (-1/2, 1/2],
    \]
    because we can add integer multiples of $b_1$ to $x$ as we choose,
    see Figure~\ref{fig:size-reduction}.

\begin{figure}
  \begin{center}
  \begin{tikzpicture}
    \draw[dotted] (-3,0) -- (4,0);
    \draw[dotted] (-3,2.3) -- (4,2.3);
    \draw (0,-1) -- (0,3) node[above] {$\langle b_1 \rangle^\bot$};
    \draw[dashed] (1,-1) -- (1,3);
    \draw[dashed] (-1,-1) -- (-1,3);
    \draw[thick,->] (0,0) -- (2,0) node[below] {$b_1$};
    \draw[thick,->] (0,0) -- (-2,0) node[below] {$-b_1$};
    \draw (0.3,0) arc[start angle=0,end angle=90,radius=0.3cm];

    \foreach \x/\y in {0/2.3,3.2/2.3,-0.8/2.3}
      \fill (\x,\y) circle[radius=2pt];
    \draw (0,2.3) node[below right] {$x'$};
    \draw (3.2,2.3) node[below right] {$x$};
    \draw (-0.8,2.3) node[above left] {$x - 2b_1$};

    \draw[->] (3.2,2.3) .. controls (2.2,2.5) .. (1.2,2.3);
    \draw[->] (1.2,2.3) .. controls (0.2,2.5) .. (-0.8,2.3);
  \end{tikzpicture}
  \end{center}
  \caption{Every projected lattice point has a pre-image whose component parallel to $b_1$
    has length at most $\|b_1\|_2 / 2$.}
  \label{fig:size-reduction}
\end{figure}

    This implies that if there were an $x' \in \pi_1(\Lambda)$
    such that a sequence in $\pi_1(\Lambda)$ has $x'$ as its limit,
    infinitely many pre-images of this sequence would have to lie
    in a ball of radius slightly larger than $\|b_1\|_2/2$
    around a pre-image of $x'$.
    This is impossible by Lemma~\ref{lemma:finitely-many-points-in-bounded-region},
    and therefore $\pi_1(\Lambda)$ is a discrete subset of $\R^d$.
  \end{proof}
  Now, $\pi_1(\Lambda)$ is a lattice of lower dimension,
  so we know that it has a basis $B' = (b_2',\ldots,b_d')$ by the induction hypothesis.
  Let $b_2, \ldots, b_d \in \Lambda$ be arbitrary pre-images of $b_2', \ldots, b_d'$, respectively.

  \begin{claim}
    $b_1, b_2, \ldots, b_d$ is a basis of $\Lambda$.
  \end{claim}
  \begin{proof}
    Clearly, $\Lambda(b_1,b_2,\ldots,b_d) \subseteq \Lambda$.
    For the reverse inclusion, let $x \in \Lambda$.
    By construction, we have
    \[
      \pi_1(x) = a_2 b_2' + \dots + a_d b_d'
    \]
    for some $a_2, \ldots, a_d \in \Z$.
    Let
    \[
      x' := a_2 b_2 + \dots + a_d b_d \in \Lambda
    \]
    be a pre-image of $\pi_1(x)$.
    Note that $x - x' \in \Ker \pi_1 = \R b_1$, that is:
    \[
      x - x' = a_1 b_1
    \]
    Furthermore, $x' - x \in \Lambda$.
    Since we chose $b_1$ to be a primitive vector of the lattice, this implies $a_1 \in \Z$,
    and we have
    \[
      x = a_1 b_1 + a_2 b_2 + \dots + a_d b_d \in \Lambda(b_1,b_2,\ldots,b_d) \qedhere
    \]
  \end{proof}
  This completes the induction step of the proof.
\end{proof}

\begin{remark}
  When we recursively choose $b_1$ as a shortest vector,
  the basis resulting from the construction of the proof is called
  an \emph{HKZ basis}, where the letters refer to Hermite, Korkine, and Zolotareff.
\end{remark}



\section*{Exercises}

\begin{enumerate}
  \item
  \begin{enumerate}[(a)]
    \item
    Let $\Lambda' \subseteq \Lambda$ be full-dimensional lattices and let $B'$ be a basis of $\Lambda'$.
    Recall that $\cP_{B'}$ is the half-open parallelepiped spanned by the columns of $B'$.
    Show:
    \[ \#(\cP_{B'} \cap \Lambda) = \frac{\det \Lambda'}{\det \Lambda} = |\Lambda : \Lambda'| \]

    \item Extend the previous statement to lattices of the same dimension that are not full-dimensional.
  \end{enumerate}
\end{enumerate}

% Copyright 2013 Nicolai Hähnle <nhaehnle@gmail.com>
%
% This work is licensed under the Creative Commons Attribution-ShareAlike 3.0
% Unported License, see http://creativecommons.org/licenses/by-sa/3.0/
%
% Among other things, this means that yes, you may take e.g. illustrations from
% the book and use them in your own work. However, (a) you must give proper
% attribution by naming me as its original author and (b) you must make your
% derivative work available under the same or similar license terms.
%
% See the Creative Commons website for the exact licensing terms.

\chapter{Reduced bases, the LLL algorithm, and approximating shortest vectors}
\label{chapter:basis-reduction-LLL}

In this chapter, we will see the lattice basis reduction algorithm due to
Lenstra, Lenstra, and Lovász~\cite{MR682664}, which we will use to approximate the shortest
vector problem.

Imagine being given a lattice basis $B \in \Q^{d \times d}$.\footnote{We
restrict our attention to rational numbers because it is unclear how to deal with arbitrary real numbers
on a computer. With some care, many but not all statements can be extended to theoretical models of computation
that allow computation with real numbers, but we will not address those questions here.}
The simplest and stupidest thing you could possibly do to approximate a shortest vector in $\Lambda(B)$
is to just return the shortest vector of this basis.
If the basis vectors are orthogonal, you will even find a shortest lattice vector
in this way.
This should be clear intuitively:
\begin{center}
  \begin{tikzpicture}
    \fill (0,0) circle (2pt) node[below left] {$0$};
    \draw[thick,->] (0,0) -- (4,0) node[below] {$b_2$};
    \draw[thick,->] (0,0) -- (0,2) node[left] {$b_1$};
  \end{tikzpicture}
\end{center}
Let us solidify our intuition formally.
For every lattice vector $x \in \Lambda(B)$,
we have
\[
  \|x\|_2^2 = \|a_1b_1 + \dots + a_db_d\|_2^2 = a_1^2 \|b_1\|_2^2 + \dots + a_d^2 \|b_d\|_2^2 \geq \|b_k\|_2^2,
\]
where $a_j \in \Z$ for all $j$,
and the last inequality holds for any $k$ with $a_k \neq 0$.
This gives us a lower bound on the length of any non-zero lattice vector,
and it just so happens that this lower bound coincides with the length of a shortest vector in the given basis.

We can extend at least part of this argument to non-orthogonal bases.
Consider the following situation:
\begin{center}
  \begin{tikzpicture}
    \fill (0,0) circle (2pt) node[below left] {$0$};
    \draw[thick,->] (0,0) -- (4,0) node[below] {$b_1$};
    \draw[thick,->] (0,0) -- (4.4,0.5) node[above] {$b_2$};
    \draw[thick,->] (0,0) -- (0.0,0.5) node[left] {$b_2^\star$};
    \draw[dotted] (0,0.5) -- (4.4,0.5);
  \end{tikzpicture}
\end{center}
Every non-zero lattice vector $x$ lies either on the line $\langle b_1 \rangle$ through $b_1$,
in which case its length is at least $\|b_1\|_2$;
or it lies on a line parallel to $\langle b_1 \rangle$.
However, every such line that contains a lattice point has distance at least $\|b_2^\star\|_2$
from the origin, where $b_2^\star = \pi_1(b_2)$ is the orthogonal projection of $b_2$
onto the orthogonal complement of $\langle b_1 \rangle$.

So $\ell := \min\{\|b_1\|_2, \|b_2^\star\|_2\}$ is a lower bound on the length $\lambda_1$
of a non-zero lattice vector.
This means that we can prove an approximation ratio of $\frac{\|b_1\|_2}{\ell}$
for the simple algorithm that just returns the first lattice basis vector.
Unfortunately, this ratio can be arbitrarily large depending on the given basis.

We will first formalize the preceding ideas using the Gram-Schmidt orthogonalization of a basis,
and will then discuss ways of finding a ``good'' basis in the sense that the resulting approximation ratio
is bounded. As the picture suggests, ``good'' in this context entails ``close to orthogonal''.



\section{Gram-Schmidt orthogonalization}

\begin{definition}
  Let $B = (b_1, \ldots, b_r) \in \R^{d \times r}$ be a lattice basis.
  Its \emph{Gram-Schmidt orthogonalization} $B^\star = (b_1^\star, \ldots, b_r^\star) \in \R^{d \times r}$
  is defined via
  \[
    b_k^\star := \pi_{k-1}(b_k)
      = b_k - \sum_{j=1}^{k-1} \underbrace{\frac{{b_k}^T b_j^\star}{{b_j^\star}^T b_j^\star}}_{=: \mu_{jk}} b_j^\star
  \]
  Recall that $\pi_{k-1} : \R^d \to \langle b_1, \ldots, b_{k-1} \rangle$ is the orthogonal projection
  onto the orthogonal complement of the space spanned by $b_1, \ldots, b_{k-1}$.
\end{definition}
\begin{lemma}
  $B^\star$ is orthogonal, that is, ${b_j^\star}^T b_k^\star = 0$ for all $j \neq k$.
\end{lemma}
\begin{proof}
  Given $j < k$, we simply compute, using the definition of $b_k^\star$:
  \begin{align*}
    {b_j^\star}^T b_k^\star
      &= {b_j^\star}^T b_k - \sum_{i=1}^{k-1} \mu_{ik} \underbrace{{b_j^\star}^T b_i^\star}_{= 0 \text{ for } j \neq i} \\
      &= {b_j^\star}^T b_k - \mu_{jk} {b_j^\star}^T b_j^\star \\
      &= {b_j^\star}^T b_k - \frac{b_k^T b_j^\star}{{b_j^\star}^T b_j^\star} {b_j^\star}^T b_j^\star = 0
  \end{align*}
  To make the second step work, we proceed by induction first over $k$ and then over $j < k$.
\end{proof}

We will often use the definition in its rearranged form:
\[
  b_k = b_k^\star + \sum_{j < k} \mu_{jk} b_j^\star
\]
We often use this set of equations for $k = 1\dots r$ in matrix form:
\[
  B = B^\star M, \text{ where } M = \begin{pmatrix}
                                      1 & \mu_{12} & \dots  & \mu_{1r} \\
                                      0 &    1     &        & \\
                                      \vdots &     & \ddots & \vdots \\
                                      0 &          &        &   1
                                    \end{pmatrix}
\]
Let us formalize the discussion at the beginning of this chapter.
\begin{lemma}
  \label{lemma:svp-lower-bound-gso}
  For all $x \in \Lambda(B) \setminus \{0\}$ we have $\|x\|_2 \geq \min_j \|b_j^\star\|_2$.
\end{lemma}
\begin{proof}
  By definition, we can write $x = a_1 b_1 + \dots + a_r b_r$,
  where $a_j \in \Z$ and there is at least one $a_j \neq 0$.
  In fact, we can write
  \[
    x = a_1 b_1 + \dots + a_k b_k
  \]
  where $k$ is the \emph{last} index with $a_k \neq 0$.
  Now, we compute a change of basis to $B^\star$:
  \begin{align*}
    x &= (a_1 + \mu_{12} a_2 + \dots + \mu_{1k} a_k) b_1^\star \\
      &+ (a_2 + \mu_{23} a_3 + \dots + \mu_{2k} a_k) b_2^\star \\
      &+ \dots \\
      &+ (a_{k-1} + \mu_{k-1,k} a_k) b_{k-1}^\star \\
      &+ a_k b_k^\star
  \end{align*}
  Using Pythagoras' theorem, it follows that
  \begin{align*}
    \|x\|_2 &\geq |a_k| \|b_k^\star\|_2 \geq \|b_k^\star\|_2 \geq \min_j \|b_j^\star\|_2 \qedhere
  \end{align*}
\end{proof}



\section{Reduced bases}

Recall the example basis from the beginning of the chapter,
and recall the observations from the proof of Lemma~\ref{lemma:every-lattice-has-basis}
about lattice points in a certain ``corridor'' around the orthogonal complement of $b_1$.
This immediately suggests that we can get a ``better'', more orthogonal basis
by replacing $b_2$ by $b_2 - b_1$ without changing the lattice:
\begin{center}
  \begin{tikzpicture}
    \fill (0,0) circle (2pt) node[below left] {$0$};
    \draw[thick,->] (0,0) -- (4,0) node[below] {$b_1$};
    \draw[thick,->] (0,0) -- (4.4,0.5) node[above] {$b_2$};
    \draw[thick,->] (0,0) -- (0.4,0.5) node[above] {$b_2 - b_1$};
    \draw[thick,->] (0,0) -- (0.0,0.5) node[left] {$b_2^\star$};
    \draw[dotted] (0,0.5) -- (4.4,0.5);
  \end{tikzpicture}
\end{center}
Visual inspection tells us that $b_2 - b_1$ is a shortest vector.
However, we can get an even more orthogonal basis:
note that this new vector is shorter than $\|b_1\|$,
and if we exchange the two basis vector, the new orthogonalization suggests another ``size reduction'' step:
\begin{center}
  \begin{tikzpicture}
    \fill (0,0) circle (2pt) node[below left] {$0$};
    \draw[thick,->] (0,0) -- (4,0) node[below] {$b_2'$};
    \draw[thick,->] (0,0) -- (0.4,0.5) node[above] {$b_1'$};
    \draw[thick,->] (0,0) -- (2.44,-1.95) node[right] {${b_2'}^\star$};
    \draw[dotted] (2.44,-1.95) -- (4,0);
    \draw[thick,->] (0,0) -- (2.4,-2) node[below] {$b_2' - 4b_1'$};
  \end{tikzpicture}
\end{center}
Surprise: the lattice has an almost exactly orthogonal basis!

The two types of operations we have performed to get to a better basis
suggest a definition of a ``good'' basis as one where these operations can no longer be performed:
\begin{definition}[$d=2$]
  \label{def:reduced-basis-2dim}
  A basis $B \in \R^{d\times 2}$ of a two-dimensional lattice is \emph{reduced} if
  \begin{enumerate}
    \item $|\mu_{12}| \leq 1/2$ (so called \emph{size-reducedness}) and
    \item $\|b_1\|_2 \leq \|b_2\|_2$.
  \end{enumerate}
\end{definition}
Let us contemplate what a reduced basis buys us.
\begin{center}
  \begin{tikzpicture}
    \fill[black!10] (-1.5,4) -- (120:3cm) arc[start angle=120, end angle=60, radius=3cm] -- (1.5,4);
    \draw[dotted] (-4,0) -- (4,0);
    \draw[dotted] (0,-1.5) -- (0,4);
    \fill (0,0) circle (2pt) node[below left] {$0$};
    \draw[thick,->] (0,0) -- (3,0) node[below right] {$b_1$};
    \draw (-30:3cm) arc[start angle=-30,end angle=210,radius=3cm];
    \draw (1.5,-1.5) -- (1.5,4);
    \draw (-1.5,-1.5) -- (-1.5,4);

    \draw[dotted] (-1.4,2.8) -- (0,2.8);
    \draw[thick,->] (0,0) -- (-1.4,2.8) node[above right] {$b_2$};
    \draw[thick,->] (0,0) -- (0,2.8) node[above right] {$b_2^\star$};
  \end{tikzpicture}
\end{center}
The definition of reduced basis implies that $b_2$ must lie in the shaded region
(or its mirror image that points down).
With this picture in mind, it is not difficult to show that $b_1$ is a shortest lattice vector.

For our purposes and with Lemma~\ref{lemma:svp-lower-bound-gso} in mind,
it is interesting to see that $b_2^\star$ can be shorter than $b_1$, but not by much:
\begin{fact}
  \label{fact:reduced-basis-2dim-size-bound}
  In a reduced basis, $\|b_2^\star\|_2 \geq \sqrt{3/4} \|b_1\|_2$.
\end{fact}
\begin{proof}
  Writing $b_2 = b_2^\star + \mu_{12} b_1^\star$, taking squares,
  and remembering that $b_1 = b_1^\star$,
  the second condition of Definition~\ref{def:reduced-basis-2dim} becomes
  \[
    \|b_1^\star\|_2^2 \leq \|b_2^\star + \mu_{12} b_1^\star\|_2^2 = \|b_2^\star\|_2^2 + \mu_{12}^2 \|b_1^\star\|_2^2
  \]
  Rearranging, we get
  \[
    \|b_2^\star\|_2^2 \geq (1 - \mu_{12}^2) \|b_1^\star\|_2^2 \geq \frac{3}{4} \|b_1^\star\|_2^2,
  \]
  where the second inequality follows from the first condition of Definition~\ref{def:reduced-basis-2dim}.
  Now we simply take square roots again to obtain the result.
\end{proof}
The proof of this fact suggests the following non-trivial generalization of
Definition~\ref{def:reduced-basis-2dim} to arbitrary dimension:
\begin{definition}
  A lattice basis is \emph{reduced} if
  \begin{enumerate}
    \item $|\mu_{ij}| \leq 1/2$ for all $i < j$ and
    \item $\|b_j^\star\|_2^2 \leq \|b_{j+1}^\star + \mu_{j,j+1} b_j^\star\|_2^2$ for all $j$ within range.
  \end{enumerate}
\end{definition}

\begin{remark}
  The definitions of reduced bases coincide for $d = 2$.
  Furthermore, the definition can be understood as saying that every ``block''
  of adjacent basis vectors is reduced in the sense that the vectors
  $\pi_{j-1}(b_j), \pi_{j-1}(b_{j+1})$ form a reduced basis of a certain $2$-dimensional lattice.
  This perspective leads to a generalized notion of
  basis reduction that has been studied by Schnorr~\cite{MR918090}.
\end{remark}

\begin{lemma}
  \label{lemma:size-bounds-reduced-basis}
  In a reduced basis $B$ of a lattice of dimension $r$, one has:
  \begin{enumerate}
    \item $\|b_{j+1}^\star\|_2 \geq \sqrt{3/4} \|b_j^\star\|_2$,
    \item $\|b_j^\star\|_2 \geq (3/4)^{(j-1)/2} \|b_1\|_2$, and
    \item $b_1$ is a $(4/3)^{(r-1)/2}$-approximation to a shortest non-zero vector of $\Lambda(B)$.
  \end{enumerate}
\end{lemma}
\begin{proof}
  The first statement is analogous to the proof of Fact~\ref{fact:reduced-basis-2dim-size-bound},
  and the second statement follows from the first by induction on $j$.

  The last statement combines the second statement with Lemma~\ref{lemma:svp-lower-bound-gso}:
  Let $x \in \Lambda(B) \setminus \{ 0 \}$.
  We can compute
  \[
    \|x\|_2 \geq \min_j \|b_j^\star\|_2 \geq \min_j (3/4)^{(j-1)/2} \|b_1\|_2 = (3/4)^{(r-1)/2} \|b_1\|_2.
  \]
  Rearranging, we get that
  \[
    \|b_1\|_2 \leq (4/3)^{(r-1)/2} \|x\|_2
  \]
  for all $x \in \Lambda(B) \setminus \{ 0 \}$.
  In particular, the inequality holds when $x$ is a shortest non-zero vector of $\Lambda(B)$,
  which is exactly the definition of an approximation.
\end{proof}

In the next sections, we will address the question
whether reduced bases always exist and whether we can compute them efficiently.



\section{The determinant of general lattices}
\label{sec:determinant-general-lattices}

Previously, we defined the determinant of a full-dimensional lattice $\Lambda$
as $\det \Lambda = |\det B|$ for some basis $B$ of $\Lambda$,
and we saw that $\det \Lambda = \vol \cP_B$.
We would like a definition of the determinant of a general lattice that coincides
with the definition we already have.
The definitions should coincide not only in the sense that they are equal for full-dimensional lattices,
but also in the sense that we get the same answer for $\Lambda \subset \R^d$
and for $\tau(\Lambda) \subset \R^r$ where $r = \dim\Lambda$ and $\tau$ is an orthogonal transformation
(that is, a transformation that preserves scalar products).

\begin{lemma}
  \label{lemma:determinant-b-star-full-dim}
  Let $B \in \R^{d \times d}$ be a basis of a full-dimensional lattice $\Lambda$.
  Then $\det \Lambda = \prod_{j=1}^d \|b_j^\star\|_2$.
\end{lemma}
\begin{proof}
  Geometrically, this is true because Gram-Schmidt orthogonalization transforms
  the fundamental parallelepiped $\cP_B$ into an \emph{orthogonal} parallelepiped $\cP_{B^\star}$
  with side lengths $\|b_j^\star\|_2$
  by a sequence of shearing operations that leave the volume unchanged.

  More formally, we can write $B = B^\star M$, where $M$ is upper triangular with $1$s on the diagonal.
  Hence $\det B = \det B^\star$. We compute
  \[
    (\det B^\star)^2 = \det {B^\star}^T B^\star = \det \begin{pmatrix}
        \|b_1^\star\|_2^2 &        & 0 \\
                          & \ddots &   \\
        0                 &        & \|b_d^\star\|_2^2
      \end{pmatrix} = \prod_j \|b_j^\star\|_2^2
  \]
  and the claim follows.
\end{proof}

\begin{definition}
  Let $\Lambda \subset \R^d$ be a lattice of dimension $r$.
  Its \emph{determinant} is given by $\det\Lambda := \sqrt{\det(B^TB)}$
  where $B \in \R^{d \times r}$ is a basis of $\Lambda$.
\end{definition}

To see that this definition is well-defined, let $B' \in \R^{d \times r}$ be another basis of $\Lambda$.
By the discussion around Lemma~\ref{lemma:basis-exchange-is-unimodular},
there is a unimodular matrix $U \in \Z^{r \times r}$ such that $B = B'U$.
We can compute
\[
  \det(B^T B) = \det(\underbrace{U^T}_{r \times r} \underbrace{{B'}^T B'}_{r \times r} \underbrace{U}_{r \times r})
    = \underbrace{\det(U)^2}_{=1} \det({B'}^T B') = \det({B'}^T B')
\]
That is, the definition of determinant does not depend on the choice of basis.

\begin{lemma}
  \label{lemma:determinant-b-star}
  Let $B \in \R^{d \times r}$ be a basis of the lattice $\Lambda$.
  Then $\det \Lambda = \prod_{j=1}^r \|b_j^\star\|_2$.
\end{lemma}
\begin{proof}
  The proof is analogous to the proof of Lemma~\ref{lemma:determinant-b-star-full-dim}.
\end{proof}

The norms $\|b_j^\star\|_2$ are invariant under orthogonal transformations,
and so this Lemma is one way to confirm that our original goal has been reached:
the new definition of the determinant of a lattice $\Lambda \subseteq \R^d$ of dimension $r$
is equal to the determinant that we obtain according to the original definition
after embedding $\Lambda$ into $\R^r$ by an orthogonal transformation.




\section{Existence of reduced bases}

Let us consider the following stylized algorithm for basis reduction:
\begin{codebox}
  \Procname{$\proc{Reduce}(B \in \Z^{d \times d})$}
  \li Perform size reduction: ensure $|\mu_{ij}| \leq 1/2$ for all $i < j$ without changing $\Lambda(B)$
  \li \If $\exists j: \|b_j^\star\|_2^2 > \|b_{j+1}^\star + \mu_{j,j+1} b_j^\star\|_2^2$
  \li \Then Swap $b_j \leftrightarrow b_{j+1}$
  \li       \Goto 1.
      \End
  \li \Return $B$
\end{codebox}
It is clear by construction that \emph{if} \proc{Reduce} terminates,
it has computed a reduced basis.
To show \emph{that} \proc{Reduce} terminates,
we analyze the potential function
\begin{align*}
  \Phi(B) &:= \prod_{j=1}^d \prod_{i=1}^j \|b_i^\star\|_2^2 \\
          &= \prod_{j=1}^d (\det \Lambda_j)^2,
\end{align*}
where $\Lambda_j := \Lambda(b_1,\ldots,b_j)$.
\begin{lemma}
  $\Phi(B) \in \N_{\geq 1}$.
\end{lemma}
\begin{proof}
  By definition, $(\det\Lambda_j)^2 = \det(B_j^T B_j)$,
  where $B_j \in \Z^{d \times j}$ contains the first $j$ columns of $B$.
  So $(\det\Lambda_j)^2$ is a positive integer, and the same holds for the product $\Phi(B)$.
\end{proof}

We leave a precise analysis of the size reduction step as an exercise.
For now, it suffices to know that size reduction performs a sequence
of elementary basis transformations of the form: replace $b_j$ by $b_j + \alpha b_i$,
where $i < j$ and $\alpha \in \Z$.\footnote{$\alpha$ is chosen so that $|\mu_{ij}| \leq 1/2$ after the transformation.}
Such transformations do not change the $\Lambda_j$ for any $j$, and therefore:

\begin{lemma}
  $\Phi(B)$ is not changed by size reduction.
\end{lemma}

\begin{lemma}
  \label{lemma:reduce-phi-decreases}
  $\Phi(B)$ strictly decreases every time line~3 of \proc{Reduce} executes.
\end{lemma}
\begin{proof}
  Let us denote by $B$ and $B'$ the lattice bases before and after the swap of $b_j$ and $b_{j+1}$,
  respectively.
  First, note that $\Lambda_i = \Lambda_i'$ for all $i \neq j$, so that
  \[
    \frac{\Phi(B')}{\Phi(B)} = \frac{(\det\Lambda_j')^2}{(\det\Lambda_j)^2}
      = \frac{\prod_{i=1}^j \|{b_i'}^\star\|_2^2}{\prod_{i=1}^j \|{b_i}^\star\|_2^2}
      = \frac{\|{b_j'}^\star\|_2^2}{\|{b_j}^\star\|_2^2}
  \]
  From the definition of Gram-Schmidt orthogonalization,
  it follows that
  \[ {b_j'}^\star = \pi_{j-1}'(b_j') = \pi_{j-1}(b_{j+1}) = b_{j+1}^\star + \mu_{j,j+1} b_j^\star \]
  Therefore,
  \[
    \Phi(B') = \frac{\|b_{j+1}^\star + \mu_{j,j+1} b_j^\star\|_2^2}{\|b_j^\star\|_2^2} \Phi(B) < \Phi(B) \qedhere
  \]
\end{proof}

To summarize: $\Phi(B)$ is integral, positive,
and decreases at least by $1$ in every iteration of \proc{Reduce}.
This means that \proc{Reduce} will eventually terminate.
We performed our analysis on lattices $\Lambda \subseteq \Z^d$,
but by a simple scaling by the lowest common denominator,
it applies equally to any rational lattice. We conclude:

\begin{theorem}
  Every rational lattice has a reduced basis.
\end{theorem}

Unfortunately, the bound on the number of iterations given by Lemma~\ref{lemma:reduce-phi-decreases}
is $\Phi(B)$ for the input lattice basis $B$.
This is a very large number: sufficient for an existence statement,
but nothing to be proud of in terms of algorithm analysis.
We do not know how to improve this analysis.
What we do know is that we can get a significantly better analysis if we slightly relax our notion of reduced basis.


\section{\texorpdfstring{$\delta$}{delta}-reduced bases}

\begin{definition}
  A lattice basis is $\delta$-reduced if
  \begin{enumerate}
    \item $|\mu_{ij}| \leq 1/2$ for all $i < j$ and
    \item $\delta \|b_j^\star\|_2^2 \leq \|b_{j+1}^\star + \mu_{j,j+1} b_j^\star\|_2^2$
  \end{enumerate}
\end{definition}
In the following, we will fix $\delta = 3/4$ for concreteness.
Such a basis is also called LLL-reduced,
after the work of Lenstra, Lenstra, and Lovász~\cite{MR682664}.

Let us modify the \proc{Reduce} accordingly:
\begin{codebox}
  \Procname{$\proc{LLL-Reduce}(B \in \Z^{d \times d})$}
  \li Perform size reduction: ensure $|\mu_{ij}| \leq 1/2$ for all $i < j$ without changing $\Lambda(B)$
  \li \If $\exists j: \frac{3}{4} \|b_j^\star\|_2^2 > \|b_{j+1}^\star + \mu_{j,j+1} b_j^\star\|_2^2$
  \li \Then Swap $b_j \leftrightarrow b_{j+1}$
  \li       \Goto 1.
      \End
  \li \Return $B$
\end{codebox}

We can now give an improved version of Lemma~\ref{lemma:reduce-phi-decreases} for this slightly modified algorithm:

\begin{lemma}
  \label{lemma:lll-reduce-phi-decreases}
  $\Phi(B)$ decreases at least by a factor $3/4$ every time line~3 of \proc{LLL-Reduce} executes.
\end{lemma}
\begin{proof}
  As in the proof of Lemma~\ref{lemma:reduce-phi-decreases},
  we call $B'$ the basis after swapping $b_j$ and $b_{j+1}$ and obtain:
  \[
    \Phi(B') = \frac{\|b_{j+1}^\star + \mu_{j,j+1} b_j^\star\|_2^2}{\|b_j^\star\|_2^2} \Phi(B) < \frac{3}{4} \Phi(B) \qedhere
  \]
\end{proof}

\begin{lemma}
  \label{lemma:lll-iterations}
  The number of iterations of \proc{LLL-Reduce} is bounded by $O(d^2 \log dM)$,
  where $M$ is an upper bound on the absolute value of the entries of the initial input basis $B$.
\end{lemma}
\begin{proof}
  Let $B^{(t)}$ be the basis after $t$ executions of line~3 of \proc{LLL-Reduce}, so that $B^{(0)} = B$.
  Then
  \begin{align}
    \label{eq:LLL-Phi-iterations}
    1 &\leq \Phi(B^{(t)}) < (3/4)^t \Phi(B)
  \end{align}
  Recall that $\Phi(B) = \prod_j \det(B_j^T B_j)$,
  where $B_j = (b_1, \ldots, b_j)$.
  Each component of $B_j^T B_j$ is integer and bounded in absolute value by $dM^2$.
  Using the Hadamard bound, we get:
  \[
    \det(B_j^T B_j) \leq \prod_{i=1}^j \| (B_j^T B_j)_i \|_2 \leq (d^2M^2)^{d/2}
  \]
  This gives us
  \[
    \Phi(B) \leq (d^2 M^2)^{d^2/2}.
  \]
  Plugging this into~\eqref{eq:LLL-Phi-iterations} and taking logarithms, we get
  \[
    t \log(4/3) < d^2 \log(dM) \qedhere
  \]
\end{proof}

\begin{theorem}
  \proc{LLL-Reduce} computes a $3/4$-reduced basis in polynomial time
  in the encoding length of the input.
\end{theorem}
\begin{proof}
  Besides Lemma~\ref{lemma:lll-iterations},
  there are two missing ingredients.
  First, we need a proper analysis of the size reduction step to argue
  that a polynomial number of basic arithmetic operations (such as addition, multiplication, and division) suffices.
  Each of those operations can be implemented in polynomial time in the encoding length of its inputs,
  so the last missing ingredient is a proof that the intermediate numbers in the algorithm
  are bounded in terms of the encoding size of the initial input basis.

  We leave those detailed proofs as an exercise.
\end{proof}

Being able to compute a $3/4$-reduced basis efficiently is all well and good.
It only becomes useful, however, because we an analogue of Lemma~\ref{lemma:size-bounds-reduced-basis} holds.

\begin{lemma}
  \label{lemma:lll-reduced-properties}
  In a $3/4$-reduced basis of a lattice of dimension $r$, one has:
  \begin{enumerate}
    \item $\| b_{j+1}^\star \|_2^2 \geq \frac{1}{2} \|b_j^\star\|_2^2$,
    \item $\| b_j^\star \|_2^2 \geq (1/2)^{j-i} \|b_i\|_2^2$ for all $i < j$, and
    \item $b_1$ is a $2^{(r-1)/2}$-approximation to a shortest non-zero vector of $\Lambda(B)$.
  \end{enumerate}
\end{lemma}
\begin{proof}
  From the definition of $3/4$-reduced basis, we get
  \[
    \frac{3}{4} \|b_j^\star \|_2^2 \leq \| b_{j+1}^\star + \mu_{j,j+1} b_j^\star \|_2^2
      = \| b_{j+1}^\star \|_2^2 + \mu_{j,j+1}^2 \|b_j^\star\|_2^2
  \]
  which we can rearrange as
  \[
    \| b_{j+1}^\star \|_2^2 \geq (3/4 - \underbrace{\mu_{j,j+1}^2}_{\leq 1/4}) \|b_j^\star\|_2^2 \geq \frac{1}{2} \|b_j^\star\|_2^2.
  \]
  From this, the second statement follows by induction,
  and the third statement is a simple application of
  Lemma~\ref{lemma:svp-lower-bound-gso} analogous to the proof of Lemma~\ref{lemma:size-bounds-reduced-basis}.
\end{proof}







\section{Nearest plane approximation for the closest vector problem}

Now that we can approximate the shortest vector problem in the $\ell_2$-norm
in polynomial time to within an exponential factor,
we can ask whether the same is possible for the closest vector problem.
Given a lattice basis $B \in \Q^{d \times d}$ and a target vector $t \in \Q^d$,
can we find a vector $x \in \Lambda(B)$ that minimizes (at least approximately) $\|x-t\|_2$?

Here is a first idea.
Given the target vector $t$, we can compute $\lambda \in \Q^d$
such that $t = B \lambda$.
We could then round each component of $\lambda$ individually
and obtain a lattice vector $B \lceil \lambda \rfloor$.
Clearly, this is going to be a very bad answer if $B$ is a bad basis.
But what if $B$ is LLL-reduced?

It turns out that the answer is positive:
When $B$ is LLL-reduced, this simple procedure is a $2^{O(d)}$-approximation algorithm~\cite{MR856638}.
However, the proof is somewhat involved.
A different, equally simple idea yields a better approximation algorithm with a simpler proof!
\begin{center}
  \begin{tikzpicture}
    \clip (-2.1,-1.1) rectangle (5.1,3.1);
    \foreach \alpha in {-1,0,...,5.1}
      \draw ($\alpha*(1.2,0)$) +(0.4,-1.1) -- +(-1.2,3.3);

    \fill (0,0) circle[radius=2pt] node[left] {$0$};

    \draw[thick,->] (0,0) -- (0.4,2.2) node[right] {$b_d$};
    \draw[thick,->] (0,0) -- (1.06,0.39) node[right] {$b_d^\star$};

    \coordinate (t) at (2.7,1.2);
    \coordinate (t') at ($(2.7,1.2) + 0.37*(1.1,0.4)$);
    \draw (t) -- (t');
    \fill (t) circle[radius=2pt] node[left] {$t$};
    \fill (t') circle[radius=2pt] node[right] {$t'$};
  \end{tikzpicture}
\end{center}
The drawing shows the parallel translates of $\Lambda' := \Lambda(b_1,\ldots,b_{d-1})$
that contain lattice points.
That is, the parallel lines correspond to $\alpha b_n + \Lambda'$ for $\alpha \in \Z$.
The idea is to project the target vector $t$ orthogonally onto the affine subspace
of the nearest such \emph{lattice hyperplane},
and then recursively approximate the closest vector problem in a $(d-1)$-dimensional lattice.
This is formalized in the algorithm \proc{NearestPlane}:
\begin{codebox}
  \Procname{$\proc{NearestPlane}(B \in \Q^{d \times r}, t \in \Q^d \cap \langle B \rangle)$}
  \li \If $r = 0$
  \li \Then \Return $0$
      \End
  \li Find $\lambda \in \Q^r$ such that $t = B \lambda$.
  \li $t' \gets t + (\lceil \lambda_r \rfloor - \lambda_r) b_r^\star$
  \li \Return $\proc{NearestPlane}((b_1,\ldots,b_{r-1}), t' - \lceil \lambda_r \rfloor b_r) + \lceil \lambda_r \rfloor b_r$
\end{codebox}
We immediately see that the returned vector $x \in \Lambda(B)$ satisfies,
for some numbers $\alpha_i$ with $|\alpha_i| \leq 1/2$:
\[
  \| x - t \|_2^2 = \|\alpha_1 b_1^\star + \dots + \alpha_r b_r^\star \|_2^2 \leq \frac{1}{4}(\|b_1^\star\|_2^2 + \dots + \|b_r^\star\|_2^2).
\]
Let $x^\star \in \Lambda(B)$ be a closest lattice vector to $t$.
Suppose that $x^\star$ does \emph{not} lie on the same lattice hyperplane as $t'$.
Then
\[
  \| x^\star - t \|_2 \geq \frac{1}{2} \|b_r^\star\|_2.
\]
On the other hand,
Lemma~\ref{lemma:lll-reduced-properties} says that if $B$ is LLL-reduced, we have
\[
  \|b_i^\star\|_2^2 \leq 2^{r-i} \|b_r^\star\|_2^2
\]
for all $i$.
These inequalities fit together:
\begin{align*}
  \| x - t \|_2^2 &\leq \frac{1}{4} (2^{r-1} + 2^{r-2} + \dots + 2 + 1) \|b_r^\star\|_2^2 \\
    &\leq 2^{r-2} \|b_r^\star\|_2^2 \\
    &\leq 2^r \| x^\star - t \|_2^2
\end{align*}
That is, if $x^\star$ is \emph{not} on the lattice hyperplane we rounded to -- a situation
that intuition tells us should be a bad one --
\proc{NearestPlane} nevertheless computes a $2^{r/2}$-approximate solution to the closest vector problem.

\begin{theorem}
  If $B$ is LLL-reduced, $\proc{NearestPlane}$ is a $2^{r/2}$-approximation algorithm
  for the closest vector problem.
\end{theorem}
\begin{proof}
  We proceed by induction on the dimension $r$ of the lattice.
  For $r = 0,1$, $\proc{NearestPlane}$ does in fact return a closest vector,
  so let us consider the case $r \geq 2$.

  We have seen that we get a $2^{r/2}$-approximation when $x^\star$ lies on a different lattice hyperplane than $t'$,
  so let us now deal with the case when $x^\star$ and $t'$ lie on the same lattice hyperplane.
  Observe that, in this case, $x^\star$ is also a closest vector to $t'$.
  We recursively compute a vector $x' \in \Lambda$ that satisfies
  \[
    \| x' - t' \|_2^2 \leq 2^{r-1} \| x^\star - t' \|_2^2
  \]
  by the induction hypothesis.
  Observe that
  \[
    \underbrace{\frac{ \| x' - t' \|_2^2 }{ \| x^\star - t' \|_2^2 }}_{\geq 1}
    \geq \frac{ \| x' - t' \|_2^2 + \| t' - t \|_2^2 }{ \| x^\star - t' \|_2^2 + \| t' - t \|_2^2 }
    = \frac{ \| x' - t \|_2^2 }{ \| x^\star - t \|_2^2 },
  \]
  where the inequality follows from the fact that a positive fraction approaches $1$
  as the same (positive) quantity is added to both numerator and denominator.
  Read the other way, the inequality yields
  \[
    \frac{ \| x' - t \|_2^2 }{ \| x^\star - t \|_2^2 }
      \leq  \frac{ \| x' - t' \|_2^2 }{ \| x^\star - t' \|_2^2 }
      \leq 2^{r-1}
    \qedhere
  \]
\end{proof}







\section*{Exercises}

\begin{enumerate}
  \item
  \begin{enumerate}[(a)]
    \item Let $b_1,b_2 \in \R^2$ be a reduced basis of $\Lambda$.
      Show: $b_1$ is a shortest vector of $\Lambda$.

    \item Show: For $d=2$,
      the algorithm \proc{Reduce} finds a reduced basis in polynomial time
      in the encoding length of the input.
  \end{enumerate}

  \item
  \begin{enumerate}[(a)]
    \item Let $\Lambda \subseteq \R^d$ be a lattice and
    let $U \subseteq \R^d$ be a subspace spanned by lattice vectors.
    Show: $\Lambda \cap U$ is a lattice, $\dim \Lambda \cap U = \dim U$.

    \item Let $\pi_{U^\bot} : \R^d \to U^\bot$ be the orthogonal projection onto the orthogonal
    complement of $U$.
    Show: $\pi_{U^\bot} (\Lambda)$ is a lattice, $\dim \pi_{U^\bot} (\Lambda) = \dim \Lambda - \dim U$.

    \item Show: $\det \Lambda = \det \Lambda \cap U \cdot \det \pi_{U^\bot} (\Lambda)$
  \end{enumerate}

  \item
    \begin{enumerate}[(a)]
      \item Show that the size reduction step of the LLL algorithm can be performed using $\poly(d)$ arithmetic operations.

        Note: You will probably want two nested loops iterating over the $\mu_{ij}$.
        Carefully evaluate the order in which you process the $\mu_{ij}$ in those loops!

      \item Show that the binary encoding sizes of all intermediate numbers used in the LLL algorithm
        are bounded by $\poly(b)$, i.e. a polynomial in the binary encoding size of the input.
    \end{enumerate}

  \item
    Let $B$ be an LLL-reduced basis of $\Lambda$.
    \begin{enumerate}[(a)]
      \item Let $x = a_1 b_1 + \dots + a_d b_d$ be a shortest vector of $\Lambda$.
        Show: $|a_j| \leq 2^{O(d)}$ for every $j = 1 \dots d$.

        Hint: You may use reverse induction where you start by showing the claim for $j = d$.

      \item Show that a shortest vector of $\Lambda$ can be computed in time $2^{O(d^2)}$.
    \end{enumerate}

  \item
    Let $B \in \Q^{d \times d}$ be an LLL-reduced basis of $\Lambda$ and let $t \in \R^d$.
    \begin{enumerate}[(a)]
      \item Let $x^\star \in \Lambda$ be a closest vector to $t$ and consider the lattice hyperplanes
        orthogonal to $b_d^\star$ (i.e., parallel to $\Lambda' := \Lambda(b_1,\ldots,b_{d-1})$.
        Let $\lambda \in \R^d$ such that $t = B \lambda$
        and let $\alpha \in \Z^d$ such that $x^\star = B \alpha$.

        Show that $|\alpha_r - \lambda_r| \leq 2^{d-2}$.

      \item Show that a closest vector to $t$ in $\Lambda$ can be computed in time $2^{O(d^2)}$.
    \end{enumerate}

  \item
    Show that given an LLL-reduced basis $B \in \Q^{d \times d}$ and target vector $t \in \Q^d$,
    the following simple rounding procedure gives a $2^{O(d)}$-approximation:
    \begin{codebox}
      \li Compute $\lambda$ such that $t = B \lambda$
      \li \Return $B \lceil \lambda \rfloor$, where $\lceil \lambda \rfloor$ is a nearest integer point.
    \end{codebox}

    Hint: \cite{MR856638}
\end{enumerate}

% Copyright 2013 Nicolai Hähnle <nhaehnle@gmail.com>
%
% This work is licensed under the Creative Commons Attribution-ShareAlike 3.0
% Unported License, see http://creativecommons.org/licenses/by-sa/3.0/
%
% Among other things, this means that yes, you may take e.g. illustrations from
% the book and use them in your own work. However, (a) you must give proper
% attribution by naming me as its original author and (b) you must make your
% derivative work available under the same or similar license terms.
%
% See the Creative Commons website for the exact licensing terms.

\chapter{The Voronoi Cell of a Lattice and its Applications}

The Voronoi diagram of a point set is defined
to provide the answer to the closest vector problem:
given a point set $P$,
the Voronoi cell of $x \in P$ is the set of points
that are closer to $x$ than to any other point in $P$.

In this chapter, we will see how the Voronoi diagram of a lattice
can be used to solve both the shortest and the closest vector problem
in single exponential time by a deterministic algorithm.
Considering how early Voronoi diagrams were first studied,
and how obvious their usefulness seems in hindsight,
it took a very long to find this algorithm
which is due to Micciancio and Voulgaris~\cite{MR2743283}.


\section{The Voronoi cell of a lattice}

\begin{definition}
  Let $\Lambda \subset \R^d$ be a lattice.
  Let $x \in \Lambda \setminus \{ 0 \}$.
  We define the open half-space
  \[
    H_x := \{ p \in \R^d ~:~ \|p\|_2 < \|x-p\|_2 \}
  \]
  and the (open) \emph{Voronoi cell} of $\Lambda$
  \[
    \cV_\Lambda := \bigcup_{x \in \Lambda\setminus\{0\}} H_x
  \]
  Figure~\ref{fig:voronoi-diagram} shows a $2$-dimensional example.
\end{definition}

\begin{figure}
  \begin{center}
    \begin{tikzpicture}
      \clip (-2.2,-2.3) rectangle (2.2,4.3);

      \fill[black!10]
        (0,1.0625) -- (0.5,0.9375) -- (0.5,-0.9375) -- (0,-1.0625) --
        (-0.5,-0.9375) -- (-0.5,0.9375) --cycle;

      \foreach \y in {-1,0,1,2}
        \foreach \x in {-3,-2,-1,0,1,2}
          \fill ($\x*(1,0) + \y*(0.5,2)$) circle[radius=2pt];

      \draw (0,0) node[below] {$0$};

      \foreach \y in {-1,0,1,2}
        \foreach \x in {-4,-3,-2,-1,0,1,2}
          \draw ($\x*(1,0) + \y*(0.5,2)$) +(0,1.0625) -- +(0.5,0.9375) -- +(0.5,-0.9375) -- +(0,-1.0625);
    \end{tikzpicture}
  \end{center}
  \caption{The Voronoi diagram of a lattice. The Voronoi cell of $0$ is shaded.}
  \label{fig:voronoi-diagram}
\end{figure}

\begin{lemma}
  If $\Lambda$ is full-dimensional, $\overline{\cV_\Lambda}$ is a polytope.

  For a general lattice $\Lambda$,
  $\overline{\cV_\Lambda} = P + \langle \Lambda \rangle^\bot$
  where $P = \overline{\cV_\Lambda} \cap \langle \Lambda \rangle$ is a polytope.
\end{lemma}
\begin{proof}
  For linearly independent vectors $x_1, \ldots, x_d \in \Lambda$,
  we have
  \[
    \cV_\Lambda \subseteq
      \underbrace{H_{x_1} \cap \dots \cap H_{x_d} \cap H_{-x_1} \cap \dots \cap H_{-x_d}}_{=: Q}.
  \]
  The right hand side $Q$ is an open parallelepiped.
  In particular, it is bounded, so $\cV_\Lambda$ is bounded.

  Let $R > 0$ such that $Q \subseteq B(0,R)$.
  Every facet of $\overline{Q}$ lies on a hyperplane of the form
  \[
    \|p\|_2 = \|x - p\|_2 \iff 2p^Tx = x^Tx
  \]
  for some $x = \pm x_j$.
  Note that $\|p\|_2 \geq \frac{1}{2} \|x\|_2$ for all points $p$ on the facet.
  On the other hand, $\|p\|_2 \leq R$ by definition of $R$,
  so that we get $\|\pm x_j\|_2 \leq 2R$ for all $j$.

  We claim that
  \[
    \cV_\Lambda = \bigcap_{x \in \Lambda\setminus\{0\} ~:~ \|x\|_2 \leq 2R} H_x
  \]
  follows.
  The inclusion from left to right is clear from the definition.
  For the inclusion from right to left,
  let $p \in H_x$ for all $x \in \Lambda\setminus\{0\}$ with $\|x\|_2 \leq 2R$.
  In particular, this implies $x \in \overline{Q}$ and therefore $\|x\|_2 \leq R$.
  This in turn implies $x \in H_y$ for all $y$ with $\|y\|_2 > 2R$.
  To summarize,
  \[
     x \in \left( \bigcap_{x \in \Lambda\setminus\{0\} ~:~ \|x\|_2 \leq 2R} H_x \right)
      \cap \left( \bigcap_{y \in \Lambda\setminus\{0\} ~:~ \|y\|_2 > 2R} H_y \right) = \cV_\Lambda
  \]
  which implies the claim.
  That is, $\cV_\Lambda$ is bounded and defined as the intersection of finitely many half-spaces,
  i.e., it is a polytope.

  The statement of the Lemma for general lattices follows
  directly from the observation that $x \in \Lambda$ is a closest lattice vector to $p$
  if and only if it is a closest lattice vector of the orthogonal projection $p'$ of $p$
  onto $\langle \Lambda \rangle$.
\end{proof}

Now that we have established that $\cV_\Lambda$ is a polytope,
a natural algorithmic representation suggests itself:

\begin{definition}
  We say that $x \in \Lambda$ is \emph{Voronoi relevant} if
  \[
    \| p \|_2 \leq \| x - p \|_2 \iff 2p^Tx \leq x^T x
  \]
  is a facet-defining inequality of $\overline{\cV_\Lambda}$.
\end{definition}

Given a representation of $\cV_\Lambda$ in terms of Voronoi relevant vectors,
simple tasks such as testing whether a point is contained in $\cV_\Lambda$
can be solved in time $N \cdot \poly(d, b)$
where $b$ is the encoding size of the coordinates
and $N$ is the number of Voronoi relevant vectors.
Our first goal in this chapter will be to bound this number $N$.
Before we can do this -- and simultaneously show how those vectors
can be found -- we need some basic tools.

\begin{lemma}
  \label{lemma:voronoi-diagram-symmetry}
  Let $x \in \frac{1}{2} \Lambda$.
  The Voronoi diagram of $\Lambda$ is symmetric with respect to $x$.
\end{lemma}
\begin{proof}
  Since the symmetry $x + u \mapsto x - u$ preserves distances,
  we only need to show that the lattice $\Lambda$ is symmetric with respect to $x$.
  Whenever $x + u \in \Lambda$, we have
  \[
    x - u = (x + u) - 2u,
  \]
  where $u \in \frac{1}{2} \Lambda$.
  That is, $x - u \in \Lambda$ as well, completing the proof.
\end{proof}

\begin{lemma}
  \label{lemma:voronoi-relevant-facet-interior}
  Let $x \in \Lambda$ be a Voronoi relevant vector.
  Then $x/2$ lies in the relative interior of the associated facet of $\cV_\Lambda$.
\end{lemma}
\begin{proof}
  The facet associated to $x$ is
  \[
    F = \cV_\Lambda \cap \{ 2p^Tx = x^Tx \}
  \]
  The point $x/2$ lies on the hyperplane defining $F$,
  and by Lemma~\ref{lemma:voronoi-diagram-symmetry},
  the entire Voronoi diagram is symmetric with respect to $x/2$.
  In particular, $F$ is symmetric with respect to $x/2$.
  Since $F$ is convex, this implies the claim.
\end{proof}

\begin{lemma}
  \label{lemma:voronoi-relevant-as-closest}
  Let $x \in \Lambda$.
  Then $x$ is Voronoi relevant if and only if
  $0$ and $x$ are the unique closest vectors to $\frac{1}{2} x$.
\end{lemma}
\begin{proof}
  Let $x$ be Voronoi relevant and
  assume that there is a third vector $y \in \Lambda$ that is equally close or closer.
  \begin{center}
    \begin{tikzpicture}
      \fill (0,0) circle[radius=2pt] node[below left] {$0$};
      \fill (3,0) circle[radius=2pt] node[below right] {$x$};
      \draw[dotted] (-0.5,0) -- (3.5,0);
      \draw[ultra thick] (1.5,-0.4) -- (1.5,0.4) node[right] {$F$};
      \draw (1.5,0) circle[radius=1.5cm];

      \fill (1.5,0) +(30:1.5) circle[radius=2pt] node[right] {$y$};
      \draw[thick] (1.5,0) +(0.188,-0.7) -- +(-0.188,0.7);
      \draw[dotted] (2.8,0.75) -- (0,0);
    \end{tikzpicture}
  \end{center}
  In this case, $x/2$ either lies on the hyperplane bounding $H_y$
  or is cut off by that hyperplane entirely.
  In any case, we have a contradiction with Lemma~\ref{lemma:voronoi-relevant-facet-interior}.

  Now suppose $0$ and $x$ are the unique closest vectors to $\frac{1}{2} x$.
  By compactness, there is a neighborhood of $\frac{1}{2} x$
  in the bounding hyperplane $\partial H_x$
  for which $0$ and $x$ are the unique closest vectors as well.
  The neighborhood is therefore contained in $\overline{\cV_\Lambda}$,
  which means that there is a facet of $\overline{\cV_\Lambda}$ in $\partial H_x$,
  i.e., $x$ is Voronoi relevant.
\end{proof}



\section{Computing Voronoi relevant vectors}

A different perspective on the statement of Lemma~\ref{lemma:voronoi-relevant-as-closest}
is the following.
The point $x/2$ lies in the refined lattice $\frac{1}{2} \Lambda$,
to which we can associate a natural \emph{parity function},
see Figure~\ref{fig:lattice-refinement-parity} for an illustration:
\begin{align*}
  \sigma : \frac{1}{2} \Lambda &\to G := (\frac{1}{2} \Lambda) / \Lambda \cong (\Z_2)^d \\
                   u &\mapsto u + \Lambda
\end{align*}%
\begin{figure}
  \begin{center}
  \begin{tikzpicture}
    \clip (-1,-1) rectangle (4,3);
    \foreach \p/\color in {{(0,0)}/black,{(0.75,0.15)}/red,{(-0.05,0.85)}/green,{(0.7,1)}/blue}
      \foreach \x in {-1,0,1,2}
        \foreach \y in {-2,-1,0,1,2}
          \path[fill=\color] ($\x*(1.5,0.3) + \y*(-0.1,1.7)$) +\p circle[radius=2pt];
  \end{tikzpicture}
  \end{center}
  \caption{A lattice $\Lambda$ in black and its refinement $\frac{1}{2} \Lambda$
    with parities indicated using different colors.}
  \label{fig:lattice-refinement-parity}
\end{figure}%
The Lemma then says that for every Voronoi relevant vector $x \in \Lambda$,
there is a vector $u \in \frac{1}{2} \Lambda$ of non-zero parity
such that $0$ and $x$ are the unique closest vectors to $u$ in $\Lambda$.
For all vectors $u' \in \frac{1}{2} \Lambda$ with the \emph{same} parity $\sigma(u') = \sigma(u)$,
the relative location of closest vectors in $\Lambda$ is the same.
This suggests the following algorithm for finding Voronoi relevant vectors:
\begin{codebox}
  \Procname{$\proc{VoronoiCell}(\Lambda)$}
  \li $X \gets \emptyset$
  \li \For $U \gets \left((\frac{1}{2} \Lambda) / \Lambda\right) \setminus \{ 0 \}$
  \li \Do $u \gets$ representative of $U$
  \li     $x \gets \proc{CVP}(\Lambda, u)$
  \li     $X \gets X \cup \{ 2(x - u), 2(u - x) \}$
      \End
  \li \Return $X$
\end{codebox}
We are purposefully imprecise about how the lattice $\Lambda$ is represented
and how the closest vector problem is to be solved.
For now, the key point is that we can find all Voronoi relevant vectors
by solving $2^d - 1$ closest vector problems:

\begin{lemma}
  \label{lemma:voronoi-cell-computation}
  The set $X$ returned by \proc{VoronoiCell} contains all Voronoi relevant vectors.
\end{lemma}
\begin{proof}
  Let $y \in \Lambda$ be a Voronoi relevant vector.
  By Lemma~\ref{lemma:voronoi-relevant-as-closest},
  the vector $y/2 \in \frac{1}{2}\Lambda \setminus \Lambda$ has exactly two closest vectors,
  $0$ and $y$.

  There is one iteration of $\proc{VoronoiCell}$ in which $\sigma(u) = \sigma(y/2)$
  or, equivalently, $u \equiv y/2 \pmod{\Lambda}$.
  In this iteration, the closest vector subroutine must return
  either $x = u - y/2$ or $x = u + y/2$, see Figure~\ref{fig:voronoi-cell-computation}.
  In both cases, the algorithm adds $y$ and $-y$ to the set $X$.
\end{proof}
\begin{figure}
  \begin{center}
  \begin{tikzpicture}
    \fill (0,0) circle[radius=2pt] node[below left] {$0$};
    \fill (2,0) circle[radius=2pt] node[below right] {$y$};
    \fill (1,0) circle[radius=2pt];
    \draw (1,0) circle[radius=1cm];
    \draw (0,0) -- (2,0);

    \fill (3,2) circle[radius=2pt] node[below] {$u$};
    \fill (4,2) circle[radius=2pt] node[below right] {$u + y/2$};
    \fill (2,2) circle[radius=2pt] node[above left] {$u - y/2$};
    \draw (3,2) circle[radius=1cm];
    \draw (2,2) -- (4,2);

    \draw[dotted] (0,0) -- (2,2);
    \draw[dotted] (1,0) -- (3,2);
    \draw[dotted] (2,0) -- (4,2);
  \end{tikzpicture}
  \end{center}
  \caption{Since $u \equiv y/2 \pmod\Lambda$, the set of closest vectors is identical after translation.}
  \label{fig:voronoi-cell-computation}
\end{figure}

\begin{example}
  The computed set $X$ may contain vectors that are not Voronoi relevant.
  Consider the case $\Lambda = \Z^2$ as shown below with its Voronoi cell.
  \begin{center}
    \begin{tikzpicture}
      \draw[fill=black!10] (-1,1) rectangle (1,-1) node[right] {$\overline{\cV_\Lambda}$};

      \foreach \x in {-1,0,1}
        \foreach \y in {-1,0,1}
          \fill ($\x*(2,0) + \y*(0,2)$) circle[radius=2pt];

      \draw (0,0) node[below] {$0$};

      \draw[fill=white] (1,1) circle[radius=2pt] node[right] {$u$};

      \draw (1,1) circle[radius=1.41cm];
    \end{tikzpicture}
  \end{center}
  During one iteration of the algorithm, $u$ will be as shown (or equivalent modulo $\Lambda$).
  There are four closest vectors to $u$, and no matter which of them is returned,
  the algorithm adds two vectors to $X$ that are not Voronoi relevant because
  the corresponding hyperplane $\partial H_x$ only induces
  a lower-dimensional face of $\overline{\cV_\Lambda}$.
\end{example}

If desired, we can detect vectors in $X$ that are not Voronoi relevant
using $O(N^2 d) = 2^{O(d)}$ arithmetic operations:
according to Lemma~\ref{lemma:voronoi-relevant-facet-interior},
it suffices to check every $x/2$ for $x \in X$ against every other $y \in X$
and compare distances.


\begin{corollary}
  \label{cor:number-of-voronoi-relevant}
  There are at most $2 \cdot (2^d - 1)$ Voronoi relevant vectors.
\end{corollary}



\section{Shortest and closest vectors via the Voronoi cell}

Now that we know how to compute the Voronoi cell,
let us see how it can be used.
First, we see that it contains a list of all shortest vectors of the lattice.

\begin{lemma}
  Let $x \in \Lambda$ be a shortest non-zero vector.
  Then $x$ is Voronoi relevant.
\end{lemma}
\begin{proof}
  Consider the ball of radius $\|x/2\|_2$ around $x/2$:
  \begin{center}
  \begin{tikzpicture}
    \fill (0,0) circle[radius=2pt] node[below left]{$0$};
    \fill (2,0) circle[radius=2pt] node[below right]{$x$};
    \fill (1,0) circle[radius=2pt] node[below]{$x/2$};

    \draw (1,0) circle[radius=1cm];
    \draw (0,0) +(-75:2cm) arc[start angle=-75,end angle=75,radius=2cm];
  \end{tikzpicture}
  \end{center}
  Since $x$ is a shortest vector, $0$ and $x$ are the unique closest vectors to $x/2$.
  By Lemma~\ref{lemma:voronoi-relevant-as-closest}, $x$ is Voronoi relevant.
\end{proof}

\begin{corollary}
  Every lattice has at most $2 \cdot (2^d - 1)$ shortest vectors.
\end{corollary}

\begin{remark}
  Let us place spheres centered at every lattice point of a radius
  such that two spheres may touch but not overlap.
  The number of spheres that touch a fixed sphere is called the \emph{kissing number}
  of the sphere packing,
  and it is equal to the number of shortest vectors in the lattice.
  Hence, the previous Corollary is an upper bound on the kissing number
  of a lattice sphere packing.

  This bound happens to be tight for $d = 2$,
  but considerably better bounds are known in higher dimension.
  A classical source on the study of sphere spackings is~\cite{MR1662447}.
\end{remark}

We can also use the Voronoi cell to compute closest vectors.
This may seem absurd at first,
because we needed to compute closest vectors in the first place to be able to
find the Voronoi relevant vectors.
However, if we need to solve \emph{many} closest vector problems,
then it may be useful to compute the Voronoi cell as a pre-processing step
to speed up future CVP computations.
We will apply this idea in Section~\ref{sec:voronoi-full-algorithm}.

For now, suppose we know the Voronoi cell $\cV$ of a lattice $\Lambda$
and are given a target vector $t \in \R^d$.
\begin{figure}
\begin{center}
  \begin{tikzpicture}
    \clip (-2.2,-2.3) rectangle (2.2,4.3);

    \foreach \y in {-1,0,1,2}
      \foreach \x in {-3,-2,-1,0,1,2}
        \fill ($\x*(1,0) + \y*(0.5,2)$) circle[radius=2pt];

    \draw (0,0) node[below] {$0$};

    \foreach \y in {-1,0,1,2}
      \foreach \x in {-4,-3,-2,-1,0,1,2}
        \draw ($\x*(1,0) + \y*(0.5,2)$) +(0,1.0625) -- +(0.5,0.9375) -- +(0.5,-0.9375) -- +(0,-1.0625);

    \coordinate (t) at (-1.1,3.7);
    \fill (t) circle[radius=2pt] node[left] {$t$};
    \draw (0,0) -- (t);
  \end{tikzpicture}
\end{center}
\caption{We find the Voronoi cell containing $t$ by tracing along the segment $[0,t]$.}
\label{fig:cvp-via-voronoi-cell-tracing}
\end{figure}
The problem of finding a closest lattice vector to $t$
is essentially the problem of finding the Voronoi cell translate that contains $t$.
We will solve that problem by following the line segment $[0,t]$,
see Figure~\ref{fig:cvp-via-voronoi-cell-tracing}.
\begin{codebox}
  \Procname{$\proc{CVP-via-Voronoi-Simple}(\Lambda, \cV_\Lambda, t)$}
  \zi $\cV_\Lambda$ is given by a list that contains all Voronoi relevant vectors
  \li $x \gets 0$
  \li \While $t \not\in x + \overline{\cV_\Lambda}$
  \li \Do Find the point $p$ where $[0,t]$ leaves $x + \cV_\Lambda$
  \li     Let $x' + \cV_\Lambda$, $x' \in \Lambda$, be the Voronoi cell entered at $p$
  \li     $x \gets x'$
      \End
  \li \Return $x$
\end{codebox}
Clearly, the algorithm returns a correct result when it terminates.
Each of the individual steps in this algorithm can be implemented in time $O(N \cdot \poly(d,b))$,
where $N$ is the size of the given list containing the Voronoi relevant vectors.
The test whether $t \in x + \overline{\cV_\Lambda}$ can be done by evaluating the $N$ linear functions
associated with the facets of $\overline{\cV_\Lambda}$.
Similarly, the point $p$ and the facet it lies on can be found by intersecting the line through $0$ and $t$
with each of the hyperplanes defining the facets of $x + \overline{\cV_\Lambda}$.

There is a subtle degenerate situation when the segment $[0,t]$ intersects a lower-dimensional
face of a cell $x + \overline{\cV_\Lambda}$.
This degeneracy can be avoided by an appropriate perturbation of the target vector $t$.

For now, let us simply assume that this degenerate situation does not happen
and continue bounding the running time of the algorithm.
Since the Voronoi cells are convex, every cell is entered at most once.
Since we only enter cells that intersect the segment $[0,t]$,
it suffices to bound the number of such cells
to bound the overall running time of the algorithm.

For every $x + \cV_\Lambda$ encountered in the algorithm,
we have $x \in [0,t] + \cV_\Lambda$
and therefore
\[
  x + \cV_\Lambda \subseteq [0,t] + 2 \cV_\Lambda
\]
Since the open cells are disjoint,
it suffices to bound the volume of $[0,t] + 2 \cV_\Lambda$.
\begin{lemma}
  Let $\lambda \in \R$ such that $t \in \lambda \cV_\Lambda$.
  Then \proc{CVP-via-Voronoi-Simple} visits at most $(\lambda+2)^d$ Voronoi cells of the lattice.
\end{lemma}
\begin{proof}
  By convexity, we have $[0,t] \subseteq \lambda \cV_\Lambda$,
  so that
  \[
    x + \cV_\Lambda \subseteq (\lambda + 2) \cV_\Lambda
  \]
  for every Voronoi cell visited by the algorithm.
  Hence, the number of such cells is bounded by
  \[
    \frac{\vol((\lambda + 2) \cV_\Lambda)}{\vol(x + \cV_\Lambda)}
      = \frac{(\lambda + 2)^d \vol \cV_\Lambda }{\vol \cV_\Lambda}
      = (\lambda + 2)^d \qedhere
  \]
\end{proof}
This seems to be a rather crude bound.
Intuitively, we expect a bound that is linear in the norm of $t$ for very large $t$.
But note that even such a bound -- and it is clear that we cannot possibly do better --
is exponential in the encoding size of $t$, which leads to an unacceptable running time.
The solution lies in geometric scaling, using two simple observations.
\begin{corollary}
  If $t \in 2\cV_\Lambda$,
  \proc{CVP-via-Voronoi-Simple} can be used to solve CVP in time $2^{O(d)} \poly(b)$.
\end{corollary}

The second observation is that the Voronoi cell of $k \Lambda$ is simply $k \cV_\Lambda$.
This means that if $t \not\in 2\cV_\Lambda$,
we can instead search for a closest vector $x'$ to $t$
in a sparser lattice $2^\kappa \cdot \Lambda$
that satisfies $t \in 2 \cdot 2^\kappa \cV_\Lambda$.
Once such a vector $x' \in \Lambda$ is found,
we can shift the problem by $-x'$,
thereby moving $t$ closer to the origin.
This is illustrated in Figure~\ref{fig:cvp-via-scaling}.

\begin{codebox}
  \Procname{$\proc{CVP-via-Voronoi-Scaling}(\Lambda, \cV_\Lambda, t)$}
  \li \If $t \in \cV_\Lambda$
  \li \Then \Return $0$
      \End
  \li $x' \gets \proc{CVP-via-Voronoi-Scaling}(2\Lambda, 2\cV_\Lambda, t)$
  \li $x'' \gets \proc{CVP-via-Voronoi-Simple}(\Lambda, \cV_\Lambda, t - x')$
  \li \Return $x' + x''$
\end{codebox}
Let $\kappa \in \N_0$ be minimal with $t \in 2^\kappa \cV_\Lambda$.
Then it is easy to see by induction over $\kappa$
that the recursion depth given input $t$ is exactly $\kappa$.

We see the correctness of the algorithm via induction over $\kappa$ as well.
It is self-evident if $\kappa = 0$, i.e., $t \in \cV_\Lambda$.
Suppose $\kappa \in \N_0$ is minimal with $t \in \cV_\Lambda$ and $\kappa \geq 1$.
Then we find, by the induction hypothesis,
$x' \in 2\Lambda$ such that $t \in x' + 2\cV_\Lambda$.
We also find $x'' \in \Lambda$ with $(t-x') \in x'' + \cV_\Lambda$,
and therefore $t \in (x' + x'') + \cV_\Lambda$,
which means the returned lattice vector is a closest vector to $t$.

Since we have $(t - x') \in 2\cV_\Lambda$,
each call of $\proc{CVP-via-Voronoi-Simple}$
takes time $2^{O(d)} \poly(b)$.
That is, the overall running of the algorithm is essentially $\kappa \cdot 2^{O(d)} \poly(b)$.
The recursion depth $\kappa$ is bounded by $\log \|t\|_2$ if $\Lambda \subseteq \Z^d$.
That is, after an appropriate scaling, we get:

\begin{theorem}
  Given a list of Voronoi relevant vectors,
  $\proc{CVP-via-Voronoi-Scaling}$ computes a closest vector to $t$ in time $2^{O(d)} \poly(b)$.
\end{theorem}


\begin{figure}
\begin{center}
  \begin{tikzpicture}
    \clip (-2.2,-2.3) rectangle (2.2,4.3);

    \foreach \y in {-1,0,1,2}
      \foreach \x in {-3,-2,-1,0,1,2}
        \fill[black!40] ($\x*(1,0) + \y*(0.5,2)$) circle[radius=2pt];

    \draw (0,0) node[below] {$0$};

    \foreach \y in {-1,0,1,2}
      \foreach \x in {-4,-3,-2,-1,0,1,2}
        \draw[black!40] ($\x*(1,0) + \y*(0.5,2)$) +(0,1.0625) -- +(0.5,0.9375) -- +(0.5,-0.9375) -- +(0,-1.0625);

    \coordinate (t) at (-1.1,3.7);
    \fill (t) circle[radius=2pt] node[left] {$t$};
    \draw (0,0) -- (t);
    \draw (-1,4) node[right] {$x'$};

    \foreach \y in {0,2}
      \foreach \x in {-2,0,2}
        \fill ($\x*(1,0) + \y*(0.5,2)$) circle[radius=2pt];

    \foreach \y in {-2,0,2}
      \foreach \x in {-4,-2,0,2}
        \draw ($\x*(1,0) + \y*(0.5,2)$) +(0,2.125) -- +(1.0,1.875) -- +(1.0,-1.875) -- +(0,-2.125);

  \end{tikzpicture}
\end{center}
  \caption{Solving the closest vector problem via \proc{CVP-via-Voronoi-Scaling}.}
  \label{fig:cvp-via-scaling}
\end{figure}



\section{A single exponential time algorithm for computing the Voronoi cell}
\label{sec:voronoi-full-algorithm}

We find ourselves in an oddly circular situation.
Given some way of computing closest vectors in $\Lambda$,
we can also compute its Voronoi relevant vectors,
and given a way of computing its Voronoi relevant vectors,
we can also computed closest vectors in $\Lambda$.
\begin{center}
  \begin{tikzpicture}
    \node[left,inner sep=3mm,draw] (cvp) at (0,0) {Closest vectors};
    \node[right,inner sep=3mm,draw] (vc) at (3,0) {Voronoi cell};

    \draw[->] ($(cvp.east) + (0,-0.1)$) -- ($(vc.west) + (0,-0.1)$);
    \draw[->] ($(vc.west) + (0,0.1)$) -- ($(cvp.east) + (0,0.1)$);
  \end{tikzpicture}
\end{center}
We will break out of this cycle by a strengthening of the nearest plane method
of Section~\ref{sec:nearest-plane-approximation}.
Given an LLL-reduced basis $B$ of $\Lambda$,
we will see how a closest vector problem in $\Lambda$
can be solved by $2^{d/2}$ closest vector problems in $\Lambda' = \Lambda(b_1,\ldots,b_{d-1})$.

Each of those closest vector problems can be solved using the Voronoi cell of $\Lambda'$,
which has to be computed only once.
The circular dependency above becomes a linear chain of dependencies
in which we compute Voronoi cells of lattices of successively higher dimension,
see Figure~\ref{fig:voronoi-cell-dimension-reduction}.

The key point of the construction is that
even though up to $2^{O(d)}$ closest vector problems are solved in each dimension,
only one Voronoi cell needs to be computed in each dimension.
Hence the overall running time remains singly exponential.
\begin{figure}
  \begin{center}
    \begin{tikzpicture}
      \node[left,inner sep=3mm,draw] (cvp1) at (0,0) {Closest vectors in $\dim = 1$};
      \node[left,inner sep=3mm,draw] (cvp2a) at (0,-1.2) {Closest vectors in $\dim = 2$};
      \node[right,inner sep=3mm,draw] (vc2) at (2.5,-1.8) {Voronoi cell in $\dim = 2$};
      \node[left,inner sep=3mm,draw] (cvp2b) at (0,-2.4) {Closest vectors in $\dim = 2$};
      \node[left,inner sep=3mm,draw] (cvp3a) at (0,-3.6) {Closest vectors in $\dim = 3$};
      \node[right,inner sep=3mm,draw] (vc3) at (2.5,-4.2) {Voronoi cell in $\dim = 3$};
      \node[left,inner sep=3mm,draw] (cvp3b) at (0,-4.8) {Closest vectors in $\dim = 3$};

      \node[left,inner sep=3mm,draw] (cvpca) at (0,-7.0) {Closest vectors in $\dim = d-1$};
      \node[right,inner sep=3mm,draw] (vcc) at (2.5,-7.6) {Voronoi cell in $\dim = d-1$};
      \node[left,inner sep=3mm,draw] (cvpcb) at (0,-8.2) {Closest vectors in $\dim = d-1$};
      \node[left,inner sep=3mm,draw] (cvpda) at (0,-9.4) {Closest vectors in $\dim = d$};
      \node[right,inner sep=3mm,draw] (vcd) at (2.5,-10.0) {Voronoi cell in $\dim = d$};

      \draw[->] (cvp1) -- (cvp2a);
      \draw[->] (cvp2a.east) -- (vc2);
      \draw[->] (vc2) -- (cvp2b.east);
      \draw[->] (cvp2b) -- (cvp3a);
      \draw[->] (cvp3a.east) -- (vc3);
      \draw[->] (vc3) -- (cvp3b.east);

      \draw[->] (cvpca.east) -- (vcc);
      \draw[->] (vcc) -- (cvpcb.east);
      \draw[->] (cvpcb) -- (cvpda);
      \draw[->] (cvpda.east) -- (vcd);
    \end{tikzpicture}
  \end{center}
  \caption{Iteratively building Voronoi cells of larger dimension.}
  \label{fig:voronoi-cell-dimension-reduction}
\end{figure}

Let us recall the nearest plane method, see Figure~\ref{fig:nearest-plane-approximation}.
We have seen that, assuming the basis $B$ is LLL-reduced,
there is a vector $x \in \Lambda$ such that
\[
  \|x-t\|_2^2 \leq 2^{d-2} \|b_d^\star\|_2^2
\]
Since $\|b_d^\star\|_2$ is the distance between adjacent lattice hyperplanes,
it follows that a closest vector to $t$ must lie,
if not on the nearest lattice hyperplane,
then on one of the $2 \cdot 2^{(d-2)/2} = 2^{d/2}$ lattice hyperplanes closest to $t$.
By projecting $t$ orthogonal onto each of those hyperplanes in turn
and solving a closest vector problem in $\Lambda'$,
we are guaranteed to find a closest vector.

\begin{theorem}
  The Voronoi relevant vectors of a lattice can be computed in time $2^{O(d)} \poly(b)$.
  As a consequence, the shortest and closest vector problems can be solved in the same asymptotic
  running time.
\end{theorem}






\section*{Exercises}

\begin{enumerate}
  \item
    \begin{enumerate}[(a)]
    \item Show that $(\frac{1}{2} \Lambda) / \Lambda \cong (\Z_2)^d$.

    \item Show that $|(\Z_2)^d| = 2^d = \frac{\det\Lambda}{\det \frac{1}{2} \Lambda}$.
    \end{enumerate}

  \item
    Define a suitable parity function on $\Lambda$ such that:
    \begin{enumerate}[(i)]
      \item A lattice vector of parity $0$ cannot define a facet of $\cV_\Lambda$.

      \item Given two vectors $x, y \in \Lambda$, $y \neq \pm x$, of the \emph{same} parity,
        show that at least one of them does not define a facet of $\cV_\Lambda$.
    \end{enumerate}
    Conclude with an alternative proof of the statement of Corollary~\ref{cor:number-of-voronoi-relevant}:
    there can be at most $2 \cdot (2^d - 1)$ Voronoi relevant vectors.
\end{enumerate}

% Copyright 2013 Nicolai Hähnle <nhaehnle@gmail.com>
%
% This work is licensed under the Creative Commons Attribution-ShareAlike 3.0
% Unported License, see http://creativecommons.org/licenses/by-sa/3.0/
%
% Among other things, this means that yes, you may take e.g. illustrations from
% the book and use them in your own work. However, (a) you must give proper
% attribution by naming me as its original author and (b) you must make your
% derivative work available under the same or similar license terms.
%
% See the Creative Commons website for the exact licensing terms.

\chapter{Dual lattices and Fourier analysis}

Consider the problem of integer programming or, more generally, lattice programming:
given a closed convex set $K$ and a lattice $\Lambda$,
decide whether there exists a lattice point $x \in K \cap \Lambda$.
A natural approach to deciding this problem
is to slice $K$ along translates of a lattice hyperplane,
analogous to the nearest-plane approach to the closest vector problem.
Each of the slices intersecting $K$ leads to an integer programming problem
of lower dimension.
\begin{center}
  \begin{tikzpicture}
    \draw[thick,fill=black!10]
      (0,0) -- (2,-1.3) -- (5,-0.2) node[below right] {$K$} arc[start angle=-50, end angle=10, radius=2cm]
      -- (3,2) -- (1,1.8) -- cycle;

    \draw (6,1.5) node[right] {$y^Tx = \alpha \in \Z$};
    \draw[->] (-0.5,0.1) -- node[left,near end] {$y$} +(0.1,-0.5);
    \clip (-1,-2) rectangle (6,2.5);
    \foreach \t in {-3,-2,-1,0,1,2}
      \draw ($(-1,0) + \t*(0,1.2)$) -- +(7,1.4);
  \end{tikzpicture}
\end{center}
These translates of lattice hyperplanes are defined by equations $y^Tx = \alpha \in \Z$,
where $y \in \R^d$ satisfies $y^Tx \in \Z$ for all $x \in \Lambda$.
This is one justification for the definition of \emph{dual lattices},
which we study in this chapter.

The running time of this particular approach to integer programming
depends strongly on how many lattice hyperplanes intersect $K$.
Fourier analysis neatly relates a lattice and its dual,
which allows us to bound the number of lattice hyperplanes that must be investigated.


\section{The dual lattice}

\begin{definition}
  Let $\Lambda \subset \R^d$ be a full dimensional lattice.
  Its \emph{dual lattice} is given by
  \[
    \Lambda^\star := \{ y \in \R^d ~:~ y^Tx \in \Z \, \forall x \in \Lambda \}
  \]
\end{definition}

\begin{lemma}
  \label{lemma:dual-basis}
  Let $B \in \R^{d \times d}$ be a basis of $\Lambda$.
  Then $\Lambda^\star = \Lambda(B^{-T})$.
  In particular, $\Lambda^\star$ is a lattice.
\end{lemma}
\begin{proof}
  Let $c_1, \ldots, c_d$ be the columns of $B^{-T}$.
  \[
    \begin{array}{|c|}
      \hline \quad c_1^T \quad  \\\hline
      \vdots \\\hline
      c_d^T \\\hline
    \end{array}
    \cdot
    \begin{array}{|c|c|c|}
      \hline  & &\\[0.5em]
      b_1 & \dots & b_d \\
      & & \\[0.5em]\hline
    \end{array}
    =
    \begin{array}{|lcr|}
      \hline 1 & & \\
       & \ddots & \\
      & & 1 \\\hline
    \end{array}
  \]
  Let $y \in \Lambda^\star$.
  Since the $c_1, \ldots, c_d$ form a basis of $\R^d$,
  we can write
  \[
    y = \alpha_1 c_1 + \dots + \alpha_d c_d,\, \alpha_j \in \R
  \]
  It suffices to show that all $c_j \in \Z$,
  which follows from
  \[
    \Z \ni y^T b_j = \alpha_1 c_1^T b_j + \dots + \alpha_d c_d^T b_j = \alpha_j.
  \]
  Now suppose $y \in \Lambda(B^{-T})$, that is,
  we can write
  \[
    y = \alpha_1 c_1 + \dots + \alpha_d c_d,\, \alpha_j \in \Z
  \]
  Let $x \in \Lambda$, that is,
  \[
    x = \beta_1 b_1 + \dots + \beta_d b_d,\, \beta_j \in \Z
  \]
  Then
  \[
    y^Tx = \alpha_1 \beta_1 + \dots + \alpha_d \beta_d \in \Z
  \]
  That is, $y^Tx \in \Z$ for all $x \in \Lambda$,
  hence $y \in \Lambda^\star$ by definition.
\end{proof}

\begin{corollary}
  Let $\Lambda$ be a full-dimensional lattice.
  \begin{enumerate}
    \item $\det \Lambda^\star = \frac{1}{\det \Lambda}$.

    \item $(\Lambda^\star)^\star = \Lambda$.

    \item $(\alpha \Lambda)^\star = \frac{1}{\alpha} \Lambda^\star$ for $\alpha > 0$.
  \end{enumerate}
\end{corollary}

Intuitively, a dense lattice has a sparse dual and vice versa.
Two aspects of this connection are formalized in the Corollary,
but it can also be seen in the lattice hyperplanes corresponding to dual lattice vectors.
For a given $y \in \Lambda^\star$,
the distance between adjacent lattice hyperplanes $y^Tx = \alpha$ and $y^Tx = \alpha + 1$
for $\alpha \in \Z$ is $1 / \|y\|_2$.
In a dense lattice, lattice hyperplane translates must lie close to each other,
which means that dual lattice vectors must be long.
We will develop more statements of this form throughout this chapter.



\section{Successive minima and covering radius}

Let us define some measures of the overall sparsity of a lattice.

\begin{definition}
  Let $\Lambda \subset \R^d$ be a full-dimensional lattice.
  The \emph{successive minima} $\lambda_1, \ldots, \lambda_d$ of $\Lambda$ are
  \[
    \lambda_j(\Lambda) := \min\{ r > 0 ~:~ \dim( \Lambda \cap B(0,r) ) \geq j \}
  \]
  where the dimension is the dimension of the linear span of the lattice points
  of norm at most $r$.

  The \emph{covering radius} of $\Lambda$ is the maximal distance from the lattice:
  \[
    \mu(\Lambda) := \max_{p \in \R^d} d(p, \Lambda)
  \]
  When the lattice is clear from the context,
  we write $\lambda_1, \ldots, \lambda_d$ and $\mu$.
  Furthermore, we write $\lambda_j^\star$ and $\mu^\star$ for the corresponding
  quantities of the dual lattice.
\end{definition}

The covering radius can be equivalently defined as the smallest radius $r$
such that the union of balls of radius $r$ around each lattice point
covers the entire space -- hence its name.

\begin{example}
  \label{example:rect-lattice-minima}
  The lattice $\Lambda := \Lambda \begin{pmatrix} 1 & 0 \\ 0 & 3 \end{pmatrix}$
  satisfies $\lambda_1 = 1$, $\lambda_2 = 3$, and $\mu = \sqrt{10}/2$.
  \begin{center}
    \begin{tikzpicture}
      \foreach \x in {-2,-1,...,2.1}
        \foreach \y in {-1,0,1}
          \fill ($\x*(0.5,0) + \y*(0,1.5)$) circle[radius=2pt];

      \draw (0,0) node[below] {$0$};

      \draw (0.25,0.75) circle[radius=0.76cm];
      \fill (0.25,0.75) circle[radius=2pt] node[right] {$p$};
    \end{tikzpicture}
  \end{center}
\end{example}

As the example shows,
$\lambda_1$ is the length of a shortest vector,
but $\lambda_2$ need not be the length of a ``second-shortest'' vector.
Instead, it is the length of a shortest vector among all vectors that are
linearly independent from a shortest vector.
In general,
we can choose linearly independent vectors $v_1, \ldots, v_d \in \Lambda$ with $\|v_j\|_2 = \lambda_j$.
Such a set of vectors is called a set of \emph{shortest independent vectors} of the lattice.
An exercise shows that such a set need \emph{not} be a basis of the lattice.

\begin{lemma}
  \label{lemma:mu-lambda-d}
  Let $\Lambda$ be a $d$-dimensional lattice.
  Then $\mu \geq \lambda_d / 2$.
\end{lemma}
\begin{proof}
  All lattice points in the interior of the ball of radius $\lambda_d$ around the origin
  lie in a hyperplane $H$.
  A point at distance $\lambda_d/2$ from $H$ has distance at least $\lambda_d/2$ to the lattice,
  see Figure~\ref{fig:mu-vs-lambda-d}.
\end{proof}
\begin{figure}
  \begin{center}
    \begin{tikzpicture}
      \fill (0,0) circle[radius=2pt];
      \draw (0,0) circle[radius=2cm];
      \draw[thick] (20:-3) -- (20:3) node[below right] {$H$};

      \draw (0,0) -- node[right] {$\lambda_d$} (110:2);
      \draw (110:1) circle[radius=1cm];
    \end{tikzpicture}
  \end{center}
  \caption{The proof that $\mu \geq \lambda_d / 2$.}
  \label{fig:mu-vs-lambda-d}
\end{figure}

The lattice $\Z$ shows that the inequality of this Lemma is tight.
Let us now relate quantities of a lattice and its dual.
The following Lemma shows that a lattice and its dual cannot be too dense at the same time.

\begin{lemma}
  \label{lemma:transference-lower-bound}
  Let $\Lambda \subset \R^d$ be a full-dimensional lattice.
  Then $\lambda_1^\star \cdot \lambda_d \geq 1$,
  and hence $\lambda_1^\star \cdot \mu \geq 1/2$.
\end{lemma}
\begin{proof}
  Let $y \in \Lambda^\star$ be a shortest vector
  and let $v_1, \ldots, v_d \in \Lambda$ be shortest independent vectors.
  Since the $v_j$ form a basis of $\R^d$,
  we must have $y^T v_k \neq 0$ for at least one $k$.
  Using the Cauchy-Schwarz inequality and the fact that $y^T v_k \in \Z$,
  we can compute
  \[
    1 \leq |y^T v_k| \leq \|y\|_2 \|v_k\|_2 = \lambda_1^\star \lambda_k \leq \lambda_1^\star \lambda_d \qedhere
  \]
\end{proof}

\begin{example}
  Consider $\Lambda = \Lambda \begin{pmatrix} 1 & 0 \\ 0 & M \end{pmatrix}$ for $M > 1$.
  We have $\Lambda^\star = \Lambda \begin{pmatrix} 1 & 0 \\ 0 & M^{-1} \end{pmatrix}$
  and therefore
  \begin{align*}
    \lambda_1 &= 1 \\
    \lambda_d &= M \\
    \lambda_1^\star &= M^{-1} \\
    \lambda_d^\star &= 1
  \end{align*}
  That is, the inequality of Lemma~\ref{lemma:transference-lower-bound} is tight,
  both when applied to $\Lambda$ and when applied to $\Lambda^\star$.
\end{example}

We would like to have an analogous statement showing
that a lattice and its dual cannot be too sparse at the same time.
That is, we would like to have an \emph{upper bound} on a product of the form $\lambda_1^\star \cdot \lambda_d$.
Some bounds of this type are shown in the exercises of this chapter,
using Minkowski's theorem and LLL-reduced bases.
We will derive a much stronger bound in Section~\ref{sec:transference-bound-banaszczyk}.



\section{Lattice width}

\begin{definition}
  Let $K \subset \R^d$ be a convex body.
  Let $\Lambda$ be a full-dimensional lattice and let $y \in \Lambda^\star$.
  The \emph{lattice width of $K$ with respect to $y$} is
  \[
    w_y(K, \Lambda) := \max_{x \in K} y^T x - \min_{x \in K} y^T x
  \]
  The \emph{lattice width of $K$} is
  \[
    w(K, \Lambda) := \min_{y \in \Lambda^\star \setminus 0} w_y(K,\Lambda)
  \]
  We write $w_y(K)$ or $w(K)$ when the lattice is clear from the context.
\end{definition}



\section{Banaszczyk's transference bound}
\label{sec:transference-bound-banaszczyk}


\section*{Exercises}

\begin{enumerate}
  \item
    Let $\Lambda \subset \R^d$ be a full-dimensional lattice.
    Show: a vector $y \in \Lambda^\star \setminus 0$ is primitive
    if and only if
    every lattice hyperplane $\{ y^T x = \alpha\}$, $\alpha \in \Z$, contains a point of $\Lambda$.

  \item
    Show that the successive minima and the covering radius of a lattice are well-defined,
    i.e. that the minimum (respectively maximum) in the definition is achieved.

  \item
    \begin{enumerate}[(a)]
    \item Show: Every $2$-dimensional lattice has a basis $(b_1, b_2)$
      in which $\|b_1\|_2 = \lambda_1$ and $\|b_2\| = \lambda_2$.

    \item
      Consider the \emph{parity lattice}
      $\Lambda := \{ x \in \Z^d ~:~ x_1 \equiv \dots \equiv x_d \pmod{2} \}$.
      Show: For $d \geq 5$, $\Lambda$ has no basis of shortest independent vectors,
      that is, there is no basis that satisfies $\|b_j\|_2 = \lambda_j$ for all $j$.
    \end{enumerate}

  \item
    Show $\lambda_1^\star \cdot \mu \leq 2^{(d-2)/2}$ using an LLL-reduced basis.

  \item
    \begin{enumerate}[(a)]
      \item Show that
        $\lambda_1 \leq 2 \left( \frac{\det \Lambda}{V_d} \right)^{1/d}$,
        where $V_d = \frac{\pi^{d/2}}{\Gamma(\frac{d}{2} + 1)}$ is the volume of a unit ball.

      \item Show: $\lambda_1 \cdot \lambda_1^\star \leq d$.
    \end{enumerate}

  \item
    Show that lattice width is invariant under linear transformations:
    Let $\Lambda$ be a lattice, $K$ be a convex body, and $f$ a linear transformation.
    Then $w(K, \Lambda) = w(f(K), f(\Lambda))$.

\end{enumerate}


% Copyright 2013 Nicolai Hähnle <nhaehnle@gmail.com>
%
% This work is licensed under the Creative Commons Attribution-ShareAlike 3.0
% Unported License, see http://creativecommons.org/licenses/by-sa/3.0/
%
% Among other things, this means that yes, you may take e.g. illustrations from
% the book and use them in your own work. However, (a) you must give proper
% attribution by naming me as its original author and (b) you must make your
% derivative work available under the same or similar license terms.
%
% See the Creative Commons website for the exact licensing terms.

\chapter{Lattice programming}


% Copyright 2013 Nicolai Hähnle <nhaehnle@gmail.com>
%
% This work is licensed under the Creative Commons Attribution-ShareAlike 3.0
% Unported License, see http://creativecommons.org/licenses/by-sa/3.0/
%
% Among other things, this means that yes, you may take e.g. illustrations from
% the book and use them in your own work. However, (a) you must give proper
% attribution by naming me as its original author and (b) you must make your
% derivative work available under the same or similar license terms.
%
% See the Creative Commons website for the exact licensing terms.

\chapter{Arbitrary Norms and Enumeration of Lattice Points}

We have seen how to find shortest and closest vectors in a lattice
with respect to the Euclidean norm in time $2^{O(d)} \cdot \poly(b)$,
where $b$ is the encoding size of the input.
What about the related problems with respect to an arbitrary norm $\|\cdot\|$?

In the decision variant of the closest vector problem,
we are given a lattice $\Lambda \subset \R^d$,
target vector $t \in \R^d$
and target distance $r > 0$,
and we have to decide whether there is a lattice point $x \in \Lambda$
with $\|x - t\| \leq r$.
This is a lattice programming problem:
decide whether the norm ball $B_{\|\cdot\|}(t,r)$ contains a lattice point.
We have seen that such problems can be solved in
$2^{O(d \log d)} \cdot \poly(b)$ arithmetic operations
and calls to an oracle describing $B_{\|\cdot\|}(t,r)$,
where $b$ is an upper bound on the encoding size of the problem,
including the logarithm of a bound on the size of $B_{\|\cdot\|}(t,r)$.
There is a gap between this running time and
the time we could previously achieve for the Euclidean norm.

We do not know how to adapt the techniques based on Voronoi cells
since Voronoi diagrams in other norms are rather complicated.
For one thing, Voronoi cells with respect to arbitrary norms are not even convex in general.

A different approach is needed.
Suppose we could somehow enumerate the lattice points in any convex body efficiently,
by an output-sensitive algorithm,
in time $(N+1)\cdot 2^{O(d)}$
where $N$ is the number of lattice points that are found.
Then a natural algorithm for solving the shortest vector problem is as follows:
Start with a lower bound $r = r_0$ on the $\|\cdot\|$-length of a shortest vector
and enumerate lattice points in the $\|\cdot\|$-ball of radius $r$ around the origin.
If any non-zero lattice points are found, output the shortest among them.
Otherwise, multiply $r$ by a constant factor and repeat.

Due to a volume argument,
the first $\|\cdot\|$-ball to contain \emph{any} non-zero lattice points
can only contain $2^{O(d)}$ of them.
Then the overall running time is still bounded by a single-exponential factor
times the logarithm of the quality of the initial estimate $r_0$.

In this chapter,
we will visit some theory on arbitrary norms,
develop a lattice point enumeration algorithm,
and use it to solve the shortest vector problem
and approximate the closest vector problem
with respect to arbitrary norms.

This approach is based on~\cite{DPV10}
and in fact works for \emph{semi-norms} and non-symmetric convex bodies.
However, we will restrict ourselves to the case of symmetric bodies
to simplify the presentation.


\section{Norms and Convex Bodies}

The unit ball
$B_{\|\cdot\|}(0,1)$
with respect to an arbitrary norm $\|\cdot\|$
is a compact convex set of positive volume.
Furthermore, it is \emph{symmetric}
in the sense that $B_{\|\cdot\|}(0,1) = -B_{\|\cdot\|}(0,1)$.

\begin{definition}
  Let $K$ be a symmetric, compact, convex set of positive volume.
  We define the norm associated to $K$ as
  \[
    \| x \|_K := \min\{ r \geq 0 ~:~ x \in r K \}
  \]
\end{definition}


\begin{lemma}
  $\|\cdot\|_K$ is a norm.
\end{lemma}
\begin{proof}
  Since $K$ has positive volume, is symmetric and convex,
  it contains a ball $B(0,\varepsilon)$ for some $\varepsilon > 0$.
  For any $x \neq 0$, this implies $x \in \frac{\|x\|_2}{\varepsilon} K$.
  So $\|x\|_K$ is well-defined for all $x \in \R^d$
  with $\|x\|_K = 0$ if and only if $x = 0$.

  In addition, we need to show $\|\lambda x\|_K = |\lambda| \cdot \|x\|_k$
  and the triangle inequality.
  By definition and using symmetry,
  \begin{align*}
    \|\lambda x\|_K
      &= \min\{ r \geq 0 ~:~ \lambda x \in r K \}
      = \min\{ r \geq 0 ~:~ x \in \frac{r}{|\lambda|} K \}
 \\ & = \min\{ |\lambda| r \geq 0 ~:~ x \in r K \}
      = |\lambda| \cdot \|x\|_K
  \end{align*}
  Let $x, y \in \R^d$.
  We have $x \in \|x\|_K \cdot K$ and $y \in \|y\|_K \cdot K$.
  Intuitively, the idea is that
  \[
    x + y \in \|x\|_K \cdot K + \|y\|_K \cdot K = (\|x\|_K + \|y\|_K) K
  \]
  and therefore $\|x+y\|_K \leq \|x\|_K + \|y\|_K$.
  The last equality of sets holds for general convex sets.
  We only need inclusion from left to right.
  Any $z \in \lambda K + \mu K$ can be written as $z = x + y$
  with $x \in \lambda K$ and $y \in \mu K$ by definition.
  \begin{align*}
    z = x + y = \lambda \cdot \underbrace{\frac{x}{\lambda}}_{\in K}
      + \mu \cdot \underbrace{\frac{y}{\mu}}_{\in K} \in (\lambda + \mu) K
  \end{align*}
  by convexity.
\end{proof}



\section{The polar of a convex body}

\begin{definition}
  Let $K \subset \R^d$ be a convex set.
  The \emph{polar} of $K$ is defined as
  \[
    K^\star := \{ y \in \R^d ~:~ y^Tx \leq 1 \,\forall x \in K \}
  \]
\end{definition}

The polar is sometimes also called the dual of a convex body.
Even though that terminology is also useful,
for example due to the connection to dual lattices that we will see,
we prefer the term polar so as not to get confused with linear programming duality.
While the definition can be written down for arbitrary convex sets,
we will see that it is most meaningful for sets that contain $0$.

\begin{example}
  The cube $[-1,+1]^d$ is the polar of the cross polytope $\conv\{\pm e_j\}$ and vice versa.
  \begin{center}
    \begin{tikzpicture}
      \draw[thick,fill=black!10] (-1,-1) -- (1,-1) -- (1,1) -- (-1,1) -- cycle;
      \draw[->] (-1.5,0) -- (1.5,0);
      \draw[->] (0,-1.5) -- (0,1.5);

      \draw[thick,fill=black!10] (3.5,0) -- (4.5,-1) -- (5.5,0) -- (4.5,1) -- cycle;
      \draw[->] (3,0) -- (6,0);
      \draw[->] (4.5,-1.5) -- (4.5,1.5);

      \draw[red] (-1.5,1) -- (1.5,1);
      \fill[red] (4.5,1) circle[radius=2pt];

      \draw[blue] (-0.5,-1.5) -- (1.5,0.5);
      \fill[blue] (5.5,-1) circle[radius=2pt];

      \draw[green!50!black] (3,0.5) -- (5,-1.5);
      \fill[green!50!black] (-1,-1) circle[radius=2pt];
    \end{tikzpicture}
  \end{center}
  Valid inequalities of $K$ correspond to points in $K^\star$ and vice versa,
  as some examples in the drawing indicate.
\end{example}

\begin{lemma}
  Let $K \subseteq \R^d$ be a convex set.
  \begin{enumerate}
    \item $K^\star$ is a closed convex set that contains the origin.
    \item If $K$ is bounded, then $0 \in \operatorname{int} K^\star$.
    \item If $0 \in \operatorname{int} K$, then $K^\star$ is bounded.
    \item $K_1 \subseteq K_2$ implies $K_1^\star \supseteq K_2^\star$.
    \item $K \subseteq (K^\star)^\star$
    \item If $K$ is closed and contains the origin, then $(K^\star)^\star = K$.
  \end{enumerate}
\end{lemma}
\begin{proof}
  For the first point, observe that $K^\star$ can be written as the (infinite) intersection
  of closed half-spaces and $0 \in K^\star$ is obvious.

  For the second point, let $K \subseteq B(0, R)$
  and let $y \in B(0, 1/R)$.
  Then for all $x \in K$, one has
  \[
    y^T x \leq \|x\|_2 \|y\|_2 \leq 1
  \]
  Hence $B(0,1/R) \subseteq K^\star$.

  For the third point, suppose $B(0,\varepsilon) \subseteq K$.
  Let $y \in K^\star$.
  Observe that $x := \varepsilon \frac{y}{\|y\|_2} \in K$,
  and so we have
  \[
    1 \geq y^Tx = \varepsilon \frac{y^Ty}{\|y\|_2} = \varepsilon \|y\|_2
  \]
  This implies $\|y\|_2 \leq 1/\varepsilon$, hence $K^\star$ is bounded.

  For the fourth point, observe that $y \in K_2^\star$
  satisfies $y^Tx \leq 1$ for all $x \in K_2$.
  In particular, it satisfies $y^Tx \leq 1$ for all $x \in K_1$,
  hence $y \in K_1^\star$.

  For the fifth point,
  let $x \in K$.
  We have $y^Tx \leq 1$ for all $y \in K^\star$ (by the definition of $K^\star$)
  and so (by the definition of $(K^\star)^\star$) we have $x \in (K^\star)^\star$.

  For the last point,
  it remains to be shown that $(K^\star)^\star \subseteq K$.
  Suppose, by way of contradiction,
  that there exists a point $z \in (K^\star)^\star$ with $z \not\in K$.
  Since $K$ is closed,
  there is a hyperplane that strictly separates $z$ from $K$.
  Formally, there exists a hyperplane with equation $a^Tx = b$
  such that $a^Tx < b$ for all $x \in K$ and $a^Tz > b$.

  Since $0 \in K$, we have $b > 0$.
  By rescaling, we can ensure $b = 1$, which implies $a \in K^\star$.
  But since $a^Tz > b = 1$, this contradicts $z \in (K^\star)^\star$.
\end{proof}

\begin{lemma}
  Let $A \in \R^{d \times d}$ be an invertible matrix.
  Then $(A K)^\star = A^{-T} K^\star$.
\end{lemma}
\begin{proof}
  This follows from a simple chain of equivalences,
  the heart of which is the fact that $y^T(Ax) = (A^T y)^T x$.
\end{proof}

The polar of a convex body is related to its primal in the same way
that a dual lattice is related to its primal as per Corollary~\ref{corollary:transformed-dual-lattice}.
Indeed, when working in a lattice $\Lambda$ with respect to a norm $\|\cdot\|_K$,
we should work in its dual with respect to the norm $\|\cdot\|_{K^\star}$.
Let us extend the covering radius and the successive minima to general norms in the natural way,
where $K$ is a symmetric convex body of positive volume:
\begin{align*}
  \lambda_j(\Lambda, K) &:= \min\{ r > 0 ~:~ \dim( \Lambda \cap B_{\|\cdot\|_K}(0,r) ) \geq j \} \\
  \mu(\Lambda, K) &:= \max_{p \in \R^d} d_{\|\cdot\|_K}(p, \Lambda)
\end{align*}
We can immediately extend some of the statements from Chapter~\ref{chapter:dual-lattices}.
\begin{lemma}
  The lattice width of $K$ is related to the length of a shortest vector in the dual by
  $w(K,\Lambda) = 2\lambda_1(\Lambda^\star, K^\star)$.
\end{lemma}
\begin{proof}
  Let $y \in \Lambda^\star$.
  \begin{align*}
    w_y(K) = \max_{x \in K} y^Tx - \min_{x \in K} y^Tx = 2 \max_{x \in K} y^Tx
  \end{align*}
  by symmetry.
  Read differently, we have
  \[
    y^Tx \leq \frac{w_y(K)}{2}
  \]
  for all $x \in K$, which implies $y \in \frac{w_y(K)}{2} K^\star$.
  On the other hand, there is an $x \in K$ that satisfies the inequality with equality,
  and hence $w_y(K) / 2$ is the smallest possible factor by which $K^\star$ must be scaled to contain $y$,
  i.e. $\|y\|_{K^\star} = w_y(K) / 2$.
  Finally,
  \[
    w(K,\Lambda) = \min_{y\in K^\star\setminus 0} w_y(K) = \min_{y\in K^\star\setminus 0} 2 \|y\|_{K^\star}
      = 2 \lambda_1(\Lambda^\star, K^\star) \qedhere
  \]
\end{proof}

\begin{lemma}[Flatness Lemma]
  Suppose one has $\lambda_1^\star(\Lambda^\star, K^\star) \cdot \mu(\Lambda, K) \leq c_d$
  for all lattices $\Lambda \subset \R^d$ and symmetric convex bodies $K \subseteq \R^d$.
  Then $w(K,\Lambda) \geq 2c_d$ implies that $p + K$ contains a lattice point for every $p \in \R^d$.
\end{lemma}
\begin{proof}
  By the previous Lemma, $2c_d \leq w(K,\Lambda) = 2 \lambda_1^\star(\Lambda^\star, K^\star)$.
  This implies $\mu(\Lambda, K) \leq 1$,
  which by definition means that for every $p \in \R^d$,
  there is an $x \in \Lambda$ with $\|p - x\|_K \leq 1$.
  But then $x \in p + K$.
\end{proof}





\section{Enumerating lattice points in an ellipsoid}

Let $\Lambda \subseteq \R^d$ be a lattice and $B = B(z,r)$ a ball.
How can we quickly enumerate the points in $B \cap \Lambda$?

Recall the notion of Voronoi diagram of a lattice from Chapter~\ref{chapter:voronoi-cell}.
We can define a graph $G = (\Lambda, E)$ by saying that
\[
  xy \in E \iff x + \overline{\cV} \text{ and } y + \overline{\cV} \text{ share a facet}
\]
Equivalently, $xy \in E$ if $y - x$ is Voronoi relevant
(recall Definition~\ref{def:voronoi-relevant}).
Intuitively, this graph is the ``dual graph'' of the Voronoi diagram.
Note that the degree of each vertex is bounded by $2\cdot (2^d - 1)$
by Corollary~\ref{cor:number-of-voronoi-relevant}.
Let $G[B]$ be the graph induced by the vertices in $B \cap \Lambda$;
see Figure~\ref{fig:voronoi-graph} for an illustration.

\begin{figure}
  \begin{center}
    \begin{tikzpicture}
      \clip (-2.2,-2.3) rectangle (2.2,4.3);

      \foreach \y in {-1,0,1,2}
        \foreach \x in {-3,-2,-1,0,1,2}
          \fill ($\x*(1,0) + \y*(0.5,2)$) circle[radius=2pt];

      \foreach \y in {-2,0,2,4}
        \draw (-2.2,\y) -- (2.2,\y);
      \foreach \x in {-4,-3,...,3.01} {
        \draw ($(-1,-4) + \x*(1,0)$) -- +(2.5,10);
        \draw ($(-1.5,6) + \x*(1,0)$) -- +(2.5,-10);
      }

      \foreach \y in {-1,0,1,2}
        \foreach \x in {-4,-3,-2,-1,0,1,2}
          \draw[help lines] ($\x*(1,0) + \y*(0.5,2)$) +(0,1.0625) -- +(0.5,0.9375) -- +(0.5,-0.9375) -- +(0,-1.0625);
    \end{tikzpicture}
    \qquad
    \begin{tikzpicture}
      \clip (-2.2,-2.3) rectangle (2.2,4.3);

      \draw[fill=black!10] (-0.2,0.8) circle[radius=1.9cm];

      \foreach \y in {-1,0,1,2}
        \foreach \x in {-3,-2,-1,0,1,2}
          \fill[help lines] ($\x*(1,0) + \y*(0.5,2)$) circle[radius=2pt];

%       \foreach \y in {-2,0,2,4}
%         \draw[help lines] (-2.2,\y) -- (2.2,\y);
%       \foreach \x in {-4,-3,...,3.01} {
%         \draw[help lines] ($(-1,-4) + \x*(1,0)$) -- +(2.5,10);
%         \draw[help lines] ($(-1.5,6) + \x*(1,0)$) -- +(2.5,-10);
%       }

      \foreach \y in {-1,0,1,2}
        \foreach \x in {-4,-3,-2,-1,0,1,2}
          \draw[help lines] ($\x*(1,0) + \y*(0.5,2)$) +(0,1.0625) -- +(0.5,0.9375) -- +(0.5,-0.9375) -- +(0,-1.0625);

      \foreach \p in {(0,0),(-1,0),(1,0),(-1.5,2),(-0.5,2),(0.5,2)}
        \fill \p circle[radius=2pt];
      \draw (-1.5,2) -- (-1,0) -- (1,0) -- (0.5,2) -- cycle;
      \draw (-1,0) -- (-0.5,2) -- (0,0) -- (0.5,2);
    \end{tikzpicture}
  \end{center}
  \caption{The dual graph of the Voronoi diagram of a lattice and the subgraph induced by a disk.}
  \label{fig:voronoi-graph}
\end{figure}

\begin{lemma}
  $G[B]$ is connected.
\end{lemma}
\begin{proof}
  Let $B = B(z,r)$ and let us briefly argue that we may assume,
  by a simple perturbation argument,
  that $z$ has a unique closest vector $x^\star \in \Lambda$.
  First, we can choose $\varepsilon > 0$ such that there is no lattice point at distance
  less than $\varepsilon$ outside $B$.
  Then, if $z$ has multiple closest vectors, we may choose one of them as $x^\star$
  and move $z$ by $\varepsilon/2$ towards it to obtain a new center $z'$.
  The ball $B(z', r + \varepsilon/2)$ contains the same lattice points as $B$,
  and $z'$ has a unique closest vector, see Figure~\ref{fig:perturbation-center-of-ball-keeps-lattice-points}.

  \begin{figure}
    \begin{center}
      \begin{tikzpicture}
        \clip (-2.2,-2.3) rectangle (2.2,4.3);

        \coordinate (z) at (-0.5,0.9375);
        \coordinate (z') at (-0.45,0.84375);

        \draw[fill=black,fill opacity=0.1] (z) circle[radius=1.3cm];
        \draw[fill=black,fill opacity=0.1] (z') circle[radius=1.4cm];

        \foreach \y in {-1,0,1,2}
          \foreach \x in {-3,-2,-1,0,1,2}
            \fill ($\x*(1,0) + \y*(0.5,2)$) circle[radius=2pt];

        \foreach \y in {-1,0,1,2}
          \foreach \x in {-4,-3,-2,-1,0,1,2}
            \draw[help lines] ($\x*(1,0) + \y*(0.5,2)$) +(0,1.0625) -- +(0.5,0.9375) -- +(0.5,-0.9375) -- +(0,-1.0625);

        \fill (z) circle[radius=2pt] node[above left] {$z$};
        \draw (z) -- +(20:1.3cm);
        \fill (z') circle[radius=2pt] node[below left] {$z'$};
        \draw (z') -- +(20:1.4cm);
      \end{tikzpicture}
    \end{center}
    \caption{Perturbing $z$ results in a ball containing the same lattice point whose center has a unique closest vector.}
    \label{fig:perturbation-center-of-ball-keeps-lattice-points}
  \end{figure}

  We will show that every $x \in \Lambda \cap B$ is connected to $x^\star$ by a path in $G[B]$.
  Choose a maximal sequence $x = x_0, x_1, \ldots, x_N \in \Lambda$ satisfying
  \begin{enumerate}
    \item $x_{j+1} - x_j$ is Voronoi relevant and
    \item $\|x_{j+1} - z\|_2 < \|x_j - z\|_2$.
  \end{enumerate}
  That is, greedily walk from $x$ along edges to get closer to $z$.
  By maximality of the sequence (equivalently: by the fact that we cannot continue greedily from $x_N$), we have
  \[
    \|x_N - z\|_2 \leq \|x_N - (z + v)\|_2
  \]
  for all Voronoi relevant vectors $v$.
  This means $x_N \in z + \overline{\cV}$,
  which implies $x_N = x^\star$ since $z$ has a unique closest vector.
  That is, we found a path from $x$ to $x^\star$ in $G[B]$.
\end{proof}

This suggests a straightforward approach to enumerating the lattice points in a ball.
First, find a closest vector to the center of the ball.
Then apply breadth-first search in $G[B]$.

There is a subtle difficulty of implementation.
Breadth-first search is well-known to take linear time when a graph is given explicitly.
However, in our case, $G[B]$ is only known implicitly.
When visiting a vertex $x \in B \cap \Lambda$,
we can easily enumerate the \emph{coordinates} of its neighbours since we know the Voronoi relevant vectors.
However, it is not so obvious how to check whether that neighbour has already been visited or not.

A simple work-around is to store all visited vertices in an efficient data structure
such as a balanced binary search tree.
Alternatively, one might visit vertices in the order of increasing distance from $x^\star$ (or $z$).
Both methods work,
although both require $O(\log |B \cap \Lambda|)$-time operations for associated checks
such as lookups in binary search trees or maintenance of a heap.

\begin{lemma}
  $\log |B \cap \Lambda| \leq \poly(b)$,
  where $b$ is the encoding size of $B$ and $\Lambda$.
\end{lemma}
\begin{proof}
  Let $B = B(z,r)$.
  Recall that we assume rational lattices for computational purposes.
  By multiplying with least common denominators, we can assume $\Lambda \subset \Z^d$ for simplicity.
  Then $\lambda_1 \geq 1$, and hence the balls of radius $1/2$ around lattice points are disjoint.
  We obtain
  \[
    |B \cap \Lambda| \cdot \vol B(0,\frac{1}{2}) \leq \vol B(z,r) = 2^d r^d \vol B(0,\frac{1}{2})
  \]
  This implies the result.
\end{proof}

That is, we can fold the running time that is required to maintain the assorted data structures
into the catch-all factor of $\poly(b)$ that the running times of all our algorithms have anyway,
and obtain:

\begin{theorem}
  There is an algorithm that,
  given a lattice $\Lambda \subset \R^d$ and an ellipsoid $E$,
  enumerates all points in $E \cap \Lambda$ in time
  $(1 + |E \cap \Lambda|) 2^{O(d)} \poly(b)$.
\end{theorem}
\begin{proof}
  We can assume that $E$ is a ball by applying a linear transformation to both $E$ and $\Lambda$.
  The rest follows from the discussion above.
\end{proof}





\section{Some examples on enumeration}

Given that we know how to enumerate lattice points in ellipsoids,
our overall strategy for enumerating lattice points in arbitrary convex bodies $K$
will be to cover $K$ by ellipsoids.

\begin{figure}
  \begin{center}
    \begin{tikzpicture}
      \foreach \y in {0,1}
        \foreach \x in {0,1,...,36.01} {
          \fill ($\y*(1,-2.4) + \x*(0.1,0.02)$) circle[radius=.5pt];
        }
      
      \draw[thick,fill=black!10] (0.5,0.05) -- (2.5,0.45) -- node[right] {$K$} (1.5,-0.35) --cycle;
    \end{tikzpicture}
  \end{center}
  \caption{There may be arbitrarily many lattice points in a neighborhood of $K$.}
  \label{fig:sparse-dense-close-to-K}
\end{figure}

Such a cover will usually not be exact, that is,
the covering ellipsoids may contain lattice points that are not in $K$.
In fact, there may be arbitrarily many lattice points just outside of $K$
even when $K$ does not contain any lattice point,
see Figure~\ref{fig:sparse-dense-close-to-K}.
We cannot hope to achieve a running time proportional to $|K \cap \Lambda|$ in this case.

\begin{definition}
  Let $K \subseteq \R^d $ be a bounded convex body and let $\Lambda \subset \R^d$ be a lattice.
  We define $G(K,\Lambda) := \max_{p \in \R^d} |(p + K) \cap \Lambda|$.
\end{definition}

Our goal will be to enumeration points in $K$ in time proportional to $G(K, \Lambda)$.

Let us now see why we cannot just enumerate points in (an approximation of)
a smallest enclosing ellipsoid.
Consider the cross-polytope $P = \conv\{\pm e_j\}$,
whose smallest enclosing ellipsoid is the unit ball.
\begin{center}
  \begin{tikzpicture}
    \draw (-1.3,0) -- (1.3,0);
    \draw (0,-1.3) -- (0,1.3);
    \draw[thick] (-1,0) -- (0,1) -- (1,0) -- (0,-1) -- cycle;
    \draw[thick] (0,0) circle[radius=1cm];
  \end{tikzpicture}
\end{center}
The cross-polytope is a union of $2^d$ unit simplices,
one for each orthant.
Hence we can compute its volume as
\[
  \vol(P) = 2^d \cdot \frac{1}{d!} \sim 2^d \frac{1}{\sqrt{2\pi d}} \left( \frac{e}{d} \right)^d
\]
where we use Stirling's approximation for the factorial.
The volume of the Euclidean unit ball is
\[
  \frac{\pi^{d/2}}{\Gamma(d/2 + 1)} \sim \pi^{d/2} \frac{1}{\sqrt{\pi d}} \left( \frac{2e}{d} \right)^{d/2}
\]
For sufficiently dense lattices, we have
$|K \cap \Lambda| \sim \frac{\vol(K)}{\det(\Lambda)}$ (see the exercises).
So let us compare volumes:
\[
  \frac{\vol(B)}{\vol(P)} \sim \frac{\pi^{d/2} \sqrt{2}}{2^{d/2}} \left( \frac{d}{e} \right)^{d/2} = 2^{\Omega(d \log d)}
\]
For a sufficiently dense lattice $\Lambda$,
the same relation holds for $|B \cap \Lambda|$ and $|P \cap \Lambda|$.
That is, enumerating the points in $B$ necessarily introduces a $2^{\Omega(d \log d)}$ factor into the running time,
which is too wasteful for our purposes.





\section*{Exercises}

\begin{enumerate}
  \item Show: $K^\star = K$ if and only if $K$ is the closed Euclidean unit ball.
  
  \item Let $K \subset \R^d$ be a convex body of positive volume.
    The goal of this exercise is to show
    that the number of lattice points in $K$ is asymptotically equal to $\vol(K) / \det(\Lambda)$
    for dense lattices.
  \begin{enumerate}[(a)]
    \item Let $\varepsilon > 0$ and consider the tiling of $\R^d$
      obtained by placing the cube $C_\varepsilon := [-\varepsilon/2, +\varepsilon/2]^d$ at
      the points $\varepsilon \Z^d$.
      Furthermore, let
      \begin{align*}
        L_\varepsilon &:= \{x \in \varepsilon \Z^d ~:~ x + C_\varepsilon \subseteq K \} \\
        U_\varepsilon &:= \{x \in \varepsilon \Z^d ~:~ (x + C_\varepsilon) \cap K \neq \emptyset \}
      \end{align*}
      Show: $|L_\varepsilon| \leq |K \cap \varepsilon \Z^d| \leq |U_\varepsilon|$ and
        $|L_\varepsilon| \leq \frac{\vol(K)}{\varepsilon^d} \leq |U_\varepsilon|$.
    
    \item Which of the inequalities in part (a) are tight?

    \item Show: There is a constant $c_K > 0$ that depends on $K$
      such that for all sufficiently small $\varepsilon > 0$ one has $|L_\varepsilon| \geq c_K \frac{\vol(K)}{\varepsilon^d}$.
      
      \emph{Hint:} Scale $K$ around an interior point.
    
    \item Show: $\lim_{\varepsilon \to 0} \frac{|U_\varepsilon|}{|L_\varepsilon|} = 1$.
    
    \emph{Hint:} You may use the fact that
      $\lim_{\varepsilon \to 0} \frac{\vol(\partial K + B(0,\varepsilon))}{2\varepsilon} = \vol_{d-1}(\partial K)$.

    \item Let $\Lambda \subset \R^d$ be a lattice.
      Show: $\lim_{\varepsilon \to 0} \frac{\vol(K)}{\varepsilon^d \det\Lambda |K \cap \varepsilon \Lambda|} = 1$.
  \end{enumerate}
\end{enumerate}

% Copyright 2013 Nicolai Hähnle <nhaehnle@gmail.com>
%
% This work is licensed under the Creative Commons Attribution-ShareAlike 3.0
% Unported License, see http://creativecommons.org/licenses/by-sa/3.0/
%
% Among other things, this means that yes, you may take e.g. illustrations from
% the book and use them in your own work. However, (a) you must give proper
% attribution by naming me as its original author and (b) you must make your
% derivative work available under the same or similar license terms.
%
% See the Creative Commons website for the exact licensing terms.

\chapter{Generating Functions and the Algebra of Polyhedra}

We have seen how to enumerate the lattice points in a convex body $K$
in time that is essentially proportional to the maximum number of lattice points
in a translate of $K$.
What if we are only interested in the \emph{number} $|K \cap \Lambda|$ of such points?

Throughout this chapter, we will work with $\Lambda = \Z^d$ -- using a linear transformation,
this is without loss of generality.
Consider the box $K = [0,M]^d$.
It contains roughly $M^d$ integer points, which is exponential in the number of bits
required to encode a description of $K$.
On the other hand, computing this number is a trivial task that can be performed in
polynomial time in the encoding length of $K$.
This leads us to suspect that \emph{computing the number} of integer points could be
exponentially faster than \emph{enumerating} all those points.

In the case of a box,
computing the number of points is a polynomial problem even when the dimension $d$ is allowed to vary.
We do not expect this to be true in general
because the problem of deciding whether $|P \cap \Z^d|$ is $0$ for an arbitrary polytope $P$ is the integer programming problem,
which is $NP$-hard.
In fact, computing $|P \cap \Z^d|$ is easily seen to be a $\# P$-hard problem.
So we should contend ourselves with looking for an algorithm that is polynomial so long as $d$ is fixed,
and with trying to reduce the dependence on $d$ of the degree of this polynomial.

The enumeration algorithms of the last chapter could be formulated with representations of convex bodies via oracles.
This is no longer the case for computing the number of integer points.
Let $N \in \N$ and consider the convex body
\[
  K_\varepsilon(a,b) := \{ (x,y) \in \R^2 ~:~ a \leq x \leq b, xy \geq N + \varepsilon, y \leq N \}
\]
The lower envelope of $K_0(a, b)$ contains an integer point if and only if $N$ has a factor between $a$ and $b$,
see Figure~\ref{fig:factoring-body}.
Consequently, for $\varepsilon \in (0,1)$ one has $|K_0 \cap \Z^2| - |K_\varepsilon \cap \Z^2| > 0$ if and only if $N$ has a factor between $a$ and $b$.
\begin{fact}
  If there were a polynomial-time algorithm for computing the number of integer points in (somewhat general) convex bodies in $\R^2$,
  then one could factor integers in polynomial time.
\end{fact}
We don't expect factoring to be a polynomial time problem,
so we cannot hope to count integer points in arbitrary convex bodies.
Instead, our goal will be a polynomial time algorithm that counts the number of integer points in \emph{polytopes}
in fixed dimension.

\begin{figure}
  \begin{center}
    \begin{tikzpicture}
      \draw[->] (0,0) -- (4.2,0);
      \draw[->] (0,0) -- (0,4.2);
      
      \draw[dashed] (0,0.666) -- (3,0.666);
      \draw[dashed] (0,2) -- (1,2);
      \draw[dashed] (0,4) -- (1,4);
      
      \draw[dashed] (1,0) -- (1,2);
      \draw[dashed] (3,0) -- (3,0.666);

      \draw (0.5,0) +(0,2pt) -- +(0,-2pt) node[below] {$1$};
      \draw (1,0) +(0,2pt) -- +(0,-2pt) node[below] {$a$};
      \draw (3,0) +(0,2pt) -- +(0,-2pt) node[below] {$b$};
      \draw (4,0) +(0,2pt) -- +(0,-2pt) node[below] {$N$};
      \draw (0,0.5) +(2pt,0) -- +(-2pt,0) node[below left] {$1$};
      \draw (0,2) +(2pt,0) -- +(-2pt,0) node[left] {$N/a$};
      \draw (0,0.666) +(2pt,0) -- +(-2pt,0) node[left] {$N/b$};
      \draw (0,4) +(2pt,0) -- +(-2pt,0) node[left] {$N$};
      
      \draw[thick,fill=black!10] plot[domain=1:3] (\x, {2 / \x}) -- (3,4) -- (1,4) -- cycle;
    \end{tikzpicture}
  \end{center}
  \caption{The body $K_0(a,b)$ encodes information about the factors of $N$.}
  \label{fig:factoring-body}
\end{figure}



\section{A simple example of generating functions}
\label{sec:simple-example-generating-functions}

Suppose we want to count the number of integer points in an interval $I := [a,b] \subset \R$
in a way that might be extendable to higher dimension.
One way of doing this is to symbolically determine the \emph{generating function}
\[
  f(I;x) = \sum_{p \in I \cap \Z} x^p
\]
and evaluate it at the point $f(I;1) = |I \cap \Z|$.
Unfortunately, $f(I;x)$ is the sum of $|I \cap \Z|$ monomials.
The memory space required for writing this sum down is therefore exponential in the encoding length of $I$.
A better approach is needed. The geometric series
\[
  \sum_{p = 0}^\infty x^p
\]
is at the same time the generating function for the unbounded interval $[0,\infty)$
and satisfies
\[
  \sum_{p = 0}^\infty x^p = \frac{1}{1 - x}
\]
where the series converges absolutely.
The following picture suggest a way to leverage the geometric series for our problem:
\begin{center}
  \begin{tikzpicture}
    \draw (-0.2,0) -- (8.2,0);
    \foreach \x in {0,0.5,...,8.01}
      \draw (\x,0) +(0,2pt) -- +(0,-2pt);
    
    \draw[very thick] (2,0) node[below] {$a$} -- node[above] {$I$} (5,0) node[below] {$b$};
    
    \draw (2,-1) +(0,2pt) -- +(0,-2pt) +(0,0) -- (8.2,-1);
    \draw (5.5,-1.5) +(0,2pt) -- +(0,-2pt) +(0,0) -- (8.2,-1.5);
    
    \draw (1,-1) node {$=$};
    \draw (1,-1.5) node {$-$};
  \end{tikzpicture}
\end{center}
The graphical ``formula'' carries over to formal Laurent series (assuming $a, b \in \Z$ for simplicity):
\[
  \sum_{p = a}^b x^p = \sum_{p = a}^\infty x^p - \sum_{p = b+1}^\infty x^p
\]
There is an open set $U \subset \C$ on which all series converge absolutely,
and so
\[
  \sum_{p = a}^b x^p = \frac{x^a}{1 - x} - \frac{x^{b + 1}}{1 - x}
\]
holds on $U$.
Since the functions involved in this equation are rational and therefore complex differentiable,
it follows that the equality must hold on all of $\C$ (except for $x = 1$, where the functions are not defined).
That is, we can write
\[
  f(I;x) = \frac{x^a}{1 - x} - \frac{x^{b + 1}}{1 - x}
\]
if we understand the equality to be up to negligible differences in the domains of the two sides of the equation.
Observe that the encoding length of the right hand side is dominated by the encodings of $a$ and $b$,
and is therefore linear in the encoding length of $I$.

The downside of this representation of $f(I;x)$ is that it is undefined at $x = 1$,
which is exactly the point where we would like to evaluate it.
However, $x = 1$ is a removable singularity.
Using the rule of de~l'Hôpital, we can compute
\[
  \lim_{x \to 1} \frac{x^a}{1 - x} - \frac{x^{b + 1}}{1 - x} = \lim_{x \to 1} \frac{x^a - x^{b+1}}{1 - x} = \lim_{x \to 1} \frac{a x^{a-1} - (b+1) x^b}{-1} = b - a + 1,
\]
which is the correct answer, as the derivation above has already shown.

Note that the graphical formula we used to derive the rational generating function for $I$ is by no means unique.
The following picture shows an alternative formula:
\begin{center}
  \begin{tikzpicture}
    \draw (-0.2,0) -- (8.2,0);
    \foreach \x in {0,0.5,...,8.01}
      \draw (\x,0) +(0,2pt) -- +(0,-2pt);
    
    \draw[very thick] (2,0) node[below] {$a$} -- node[above] {$I$} (5,0) node[below] {$b$};
    
    \draw (2,-1) +(0,2pt) -- +(0,-2pt) +(0,0) -- (8.2,-1);
    \draw (-0.2,-1.5) -- (5,-1.5)  +(0,2pt) -- +(0,-2pt);
    \draw (-0.2,-2) -- (8.2,-2);
    
    \draw (-1,-1) node {$=$};
    \draw (-1,-1.5) node {$+$};
    \draw (-1,-2) node {$-$};
  \end{tikzpicture}
\end{center}
Again, this carries over to formal Laurent series:
\[
  \sum_{p = a}^b x^p = \sum_{p = a}^\infty x^p + \sum_{p = -\infty}^b x^p - \sum_{p = -\infty}^\infty x^p
\]
There are two problems:
\begin{itemize}
  \item The first two series on the right hand side are geometric series for which a rational function expression is known.
    However, their regions of absolute convergence are disjoint.
  
  \item The last series does not converge anywhere, and it is unclear what rational function expression could be used in its place.
\end{itemize}
Over the course of this chapter,
we will see that the first problem turns out not to be a problem after all,
and that the last, non-converging series, magically disappears.
One finds that
\[
  f(I;x) = \frac{x^a}{1-x} + \frac{x^b}{1-x^{-1}}
\]
where, again, the equality should be understood modulo negligible differences in the functions' domains.
For now, the reader may want to convince themselves that the equation holds in this particular example,
for example by manual comparison to the rational function obtained previously.

The remainder of this chapter generalizes the approach we have just laid out to higher dimension,
exploring some rather fascinating algebraic structures on the way.



\section{Cones and triangulations}

\begin{definition}
  \begin{enumerate}[(a)]
    \item A non-empty convex set $K \subseteq \R^d$ is a \emph{cone}
      if $p \in K$ and $\lambda \geq 0$ imply $\lambda p \in K$.
  
    \item A \emph{conic combination} of vectors $u_1, \ldots, u_n \in \R^d$ is
      a linear combination with non-negative coefficients.
  
    \item Given $U \subset \R^d$, the set $\cone(U)$ is the set of conic combinations of vectors in $U$.
  
    \item A cone $K$ is \emph{polyhedral} if $K$ is a polyhedron.
  \end{enumerate}
\end{definition}

Note that cones are closed under taking conic combinations,
and $\cone{U}$ is the smallest cone containing $U$.

\begin{definition}
  Let $P$ be a non-empty polyhedron.
  The \emph{lineality space} of $P$ is
  \[
    L(P) := \{ v \in \R^d ~:~ \forall x\in P \,\forall \lambda \in \R:~ x + \lambda v \in P  \}
  \]
  If $L(P) = 0$, $P$ is called \emph{pointed}.
\end{definition}

It is not difficult to check that $L(P)$ is a linear subspace of $\R^d$.
Intuitively, $L(P)$ contains the directions of lines contained in $P$.
Polyhedra containing lines are somewhat unusual in the sense
that if one thinks of polyhedra, one usually thinks of polyhedra that have vertices,
and such polyhedra \emph{do not} contain lines.
That is, their lineality space is $0$.
However, their properties are crucial for the development of compact rational generating functions.

The following facts tie cones, polarity, and lineality spaces together.
Their proofs are left as an exercise.

\begin{lemma}
  \label{lemma:cone-polars}
  Let $K \subseteq \R^d$ be a closed cone.
  \begin{enumerate}[(a)]
    \item $K^\star = \{ y \in \R^d ~:~ y^Tx \leq 0 \,\forall\, x\in K \}$.
    \item $K^\star$ is a cone and $(K^\star)^\star = K$.
    \item $d = \dim L(K) + \dim K^\star$.
  \end{enumerate}
\end{lemma}

\begin{definition}
  A cone $K$ is \emph{simplicial} if $K = \cone\{ u_1, \ldots, u_k \}$ for linearly independent vectors $u_j \in \R^d$.
\end{definition}

\begin{definition}
  Let $K \subset \R^d$ be a full-dimensional simplicial cone,
  $K = \cone\{ u_1, \ldots, u_d \}$
  with $u_j \in \Z^d$ primitive integer vectors.
  Then the \emph{determinant} of $K$ is
  \[
    \det(K) := |\det(u_1,\ldots,u_d)|
  \]
  A \emph{unimodular} cone is a full-dimensional simplicial cone with determinant $1$.
\end{definition}

Again, the proofs of the following facts about simplicial and unimodular cones are left as an exercise.

\begin{lemma}
  \label{lemma:simplicial-cone-polars}
  Let $K = \cone\{u_1,\ldots,u_d\} \subset \R^d$ be a full-dimensional simplicial cone.
  \begin{enumerate}[(a)]
    \item Let $A = (u_1, \ldots, u_d)$. Then $K = \{ p \in \R^d ~:~ -A^{-1} p \leq 0 \}$.
    \item $K^\star$ is simplicial.
    \item $K$ is unimodular if and only if $K^\star$ is unimodular.
  \end{enumerate}
\end{lemma}

\begin{figure}
  \begin{center}
    \begin{tikzpicture}
      \clip (-3.1,-3.1) rectangle (3.1,3.1);
      
      \draw[thick,fill=black!10] (-0.5,5) -- (0,0) -- (6,5);
      \draw[thick,fill=black!10] (-10,-1) -- (0,0) -- (10,-12);
      \draw (2,2.5) node {$K$};
      \draw (-2.5,-2.5) node {$K^\star$};
    \end{tikzpicture}
  \end{center}
  \caption{A simplicial cone and its polar.}
\end{figure}


Working with simplicial cones is desirable due to their simple combinatorial structure.
For this reason, we want to \emph{triangulate} more complicated polyhedral cones into simplicial cones.
We start by defining the notion of triangulation for polytopes,
see Figure~\ref{fig:triangulation-example}.

\begin{definition}
  Let $P \subset \R^d$ be a polytope.
  A set of simplices $\Delta_1, \ldots, \Delta_n$ is a \emph{triangulation} of $P$ if
  \begin{enumerate}
    \item $P = \bigcup_j \Delta_j$ and
    \item for every $i \neq j$, $\Delta_i \cap \Delta_j$ is a face of both $\Delta_i$ and $\Delta_j$ (possibly the empty face).
  \end{enumerate}
\end{definition}

\begin{figure}
  \begin{center}
    \begin{tikzpicture}
      \coordinate (n1) at (0,0);
      \coordinate (n2) at (2,0);
      \coordinate (n3) at (2.3,1.5);
      \coordinate (n4) at (1.8,2.5);
      \coordinate (n5) at (0.5,2.5);
      \coordinate (n6) at (-0.2,1);
      
      \draw[thick,fill=black!10] (n1) -- (n2) -- (n3) -- (n4) -- (n5) -- (n6) -- cycle;
      \draw[thick] (n1) -- (n3);
      \draw[thick] (n1) -- (n4);
      \draw[thick] (n4) -- (n6);
    \end{tikzpicture} \qquad
    \begin{tikzpicture}
      \coordinate (n1) at (0,0);
      \coordinate (n2) at (2,0);
      \coordinate (n3) at (2.3,1.5);
      \coordinate (n4) at (1.8,2.5);
      \coordinate (n5) at (0.5,2.5);
      \coordinate (n6) at (-0.2,1);
      \coordinate (t0) at (0.9,0);
      \coordinate (t1) at (1,1.1);
      
      \draw[thick,fill=black!10] (n1) -- (n2) -- (n3) -- (n4) -- (n5) -- (n6) -- cycle;
      \draw[thick] (n1) -- (t1);
      \draw[thick] (n6) -- (t1);
      \draw[thick] (n4) -- (t1);
      \draw[thick] (t0) -- (n3);
      \draw[thick] (n4) -- (n6);
      \draw[thick] (t0) -- (t1);
      \draw[thick] (t0) -- (n4);
    \end{tikzpicture} \qquad
    \begin{tikzpicture}
      \coordinate (n1) at (0,0);
      \coordinate (n2) at (2,0);
      \coordinate (n3) at (2.3,1.5);
      \coordinate (n4) at (1.8,2.5);
      \coordinate (n5) at (0.5,2.5);
      \coordinate (n6) at (-0.2,1);
      \coordinate (t0) at (1.15,0.75);
      
      \draw[thick,fill=black!10] (n1) -- (n2) -- (n3) -- (n4) -- (n5) -- (n6) -- cycle;
      \draw[thick] (n1) -- (n3);
      \draw[thick] (n1) -- (n4);
      \draw[thick] (n4) -- (n6);
      \draw[thick] (n2) -- (t0);
    \end{tikzpicture}
  \end{center}
  \caption{The first two pictures show triangulations. The last picture does not satisfy the second condition of the definition.}
  \label{fig:triangulation-example}
\end{figure}

\begin{theorem}
  \label{thm:polytope-triangulation}
  Let $P = \conv\{ v_1, \ldots, v_n \} \subset \R^d$ be a full-dimensional polytope.
  There is a triangulation $\Delta_1, \ldots, \Delta_m$ of $P$
  such that each simplex $\Delta_j$ is spanned by some subset of $d+1$ vectors from the set $\{ v_j \}_j$.

  If $P$ is rational, such a triangulation can be computed in polynomial time if $d$ is fixed.
\end{theorem}
\begin{figure}
  \begin{center}
    \begin{tikzpicture}
      \draw[thick,fill=black!10] (0,0) -- (2,0) -- (3.4,0.8) -- (1.4,0.8) -- cycle;
      
      \draw[dotted] (0,0) -- (0,1.5);
      \draw[dotted] (2,0) -- (2,1.5);
      \draw[dotted] (3.4,0.8) -- (3.4,3.0);
      \draw[dotted] (1.4,0.8) -- (1.4,2.3);

      \draw[thick,fill=black!10] (0,1.5) -- (2,1.5) -- (3.4,3.0) -- (1.4,2.3) -- cycle;
      \draw[thick,dashed] (2,1.5) -- (1.4,2.3);
      \draw[thick] (0,1.5) -- (3.4,3.0);
      
      \draw[thick,dashed] (2,0) -- (1.4,0.8);
      
      \draw (3.4,0.4) node {$P$};
      \draw (3.4,2.25) node {$P'$};
    \end{tikzpicture}
  \end{center}
  \caption{Triangulate a polytope $P \subset \R^2$ by the projection of the lower envelope of a lifting.}
  \label{fig:triangulation-proof}
\end{figure}
\begin{proof}
  The idea is to lift the vertices of $P$ into $\R^{d+1}$ in such a way
  that essentially every facet of their convex hull $P'$ is a simplex.
  The projection of the lower envelope of $P'$ will then be a triangulation of $P$,
  see Figure~\ref{fig:triangulation-proof}.

  Let us choose $t_1, \ldots, t_n \in \R$ in a manner that is to be described later
  and let $v_i' := (v_i, t_i) \in \R^{d + 1}$.
  Their convex hull
  \[
    P' := \conv\{v_1',\ldots,v_n'\}
  \]
  is a polytope with $\pi(P') = P$,
  where $\pi:\R^{d+1} \to \R^d$ is the projection onto the first $d$ coordinates.

  For every $x \in P$, let
  \[
    \ell(x) := \min\{ t ~:~ (x,t) \in P' \}
  \]
  The point $(x, \ell(x))$ lies in a facet of $P'$
  whose defining inequality $a^Tx + \alpha t \geq \beta$ satisfies $\alpha > 0$.\footnote{This follows
  from the minimality of $\ell(x)$.}
  Let $F_1,\ldots,F_m$ be the set of facets of $P'$ whose defining inequality satisfies $\alpha > 0$
  and let $\Delta_j := \pi(F_j)$.
  We have already shown that
  \[
    P = \bigcup_j \Delta_j
  \]
  Since $\pi$ is injective on the $F_j$, the $\Delta_j$ are full-dimensional
  and for $i \neq j$, $\Delta_i \cap \Delta_j$ is either empty or a face of both $\Delta_i$ and $\Delta_j$,
  since the same property holds for the $F_j$ by basic combinatorics of polyhedra.

  It remains to be shown that the $\Delta_j$ are simplices.
  Standard perturbation arguments show that the $t_i$ can be chosen such that no $d+2$ of the $v_i'$ lie in a hyperplane that does not contain $\ker \pi$.

  Let us show an argument via the polynomial method in detail.
  Consider a subset of $d+1$ affinely independent vertices of $P$, say $v_1, \ldots, v_{d+1}$.
  Their liftings $v_1', \ldots, v_{d+1}'$ will span a hyperplane $H \subset \R^{d+1}$ that does not contain $\ker \pi$.
  Any normal vector $u' = (u,\nu)$ of $H$ satisfies
  \[
    (v_i' - v_{d+1}')^T u' = 0 \text{ for all } i = 1 \ldots d
  \]
  and we can normalize it by requiring
  \[
    \nu = 1.
  \]
  We have obtained a linear system for $u'$ with a square, invertible\footnote{%
  Since we assumed $v_1, \ldots, v_{d+1}$ to be affinely independent, the vectors $v_j - v_{d+1}$ are linearly indpendent.}
  coefficient matrix of the form:
  \[
    \begin{pmatrix}
      v_1^T - v_{d+1}^T & t_1 - t_{d+1} \\
      \vdots & \vdots \\
      v_d^T - v_{d+1}^T & t_d - t_{d+1} \\
      0 & 1
    \end{pmatrix}
  \]
  Using Cramer's rule, we can conclude that each component of $u$ is a multilinear function of the $t_i$.

  Our goal is to choose the $t$ in such a way that every set of $d+1$ affinely independent vertices ends up with a different vector $u$.
  Let
  \[
    t(\lambda) := (\lambda^1, \ldots, \lambda^n)
  \]
  so that every component of every $u$ is a polynomial in $\lambda$.
  Moreover, if we fix two different sets of $d+1$ affinely independent vertices
  and call the corresponding normal vectors $u$ and $\hat u$,
  then the components of the vector $u - \hat u$ are polynomials in $\lambda$,
  at least one of which is non-zero.\footnote{To see this,
  observe that the exponents of the monomials $\lambda^i$ appearing in the polynomials correspond to the indices of vertices.
  Since different sets of vertices are used to define $u$ and $\hat u$,
  they also contain different sets of monomials, which cannot all cancel.}
  That is, there are finitely many values of $\lambda$ for which $u$ and $\hat u$ are equal.

  Taking all pairs of sets of vertices into account,
  there are only finitely many values of $\lambda$
  for which the liftings of two different affinely independent sets of $d+1$ vertices lie in the same hyperplane.
  For all other values of $\lambda$, the facets on the lower envelope of the resulting $P'$ are guaranteed to be simplices.

  Finally, let us remark that the convex hull of a set of points can be computed in polynomial time when the dimension $d$ is fixed,
  and the number of ``bad'' values for $\lambda$ is bounded by $n \binom{n}{d+1}^2$, which is a polynomial in $n$ for fixed $d$.
  Hence all steps of the proof can be made algorithmic, and the resulting algorithm runs in polynomial time
  when the dimension $d$ is fixed.
\end{proof}

\begin{corollary}
  Let $C = \cone\{ u_1, \ldots, u_n \} \subset \R^d$ be a full-dimensional pointed cone.
  Then there exist full-dimensional simplicial cones $C_1, \ldots, C_m \subset \R^d$ spanned by the $u_i$
  such that
  \begin{enumerate}
    \item $C = \bigcup_i C_i$ and
    \item for all $i \neq j$, $C_i \cap C_j$ is a face of both $C_i$ and $C_j$ (possibly the face $\{ 0 \}$).
  \end{enumerate}
  If $C$ is a rational cone, this triangulation can be computed in polynomial time for fixed dimension $d$.
\end{corollary}
\begin{figure}
  \begin{center}
    \begin{tikzpicture}
      \draw[thick,dotted] (1,-1.5) -- (1.52,1.5);

      \draw[thick,fill=black!10] (0,0) -- (2,0) -- (3.4,0.8) -- (1.4,0.8) -- cycle;
      \draw[thick,dashed] (2,0) -- (1.4,0.8);

      \draw[thick] (1,-1.5) -- (-1,1.5);
      \draw[thick] (1,-1.5) -- (3,1.5);
      \draw[thick] (1,-1.5) -- (4.13,1.5);
    \end{tikzpicture}
  \end{center}
  \caption{The triangulation of a pointed cone can be obtained from a triangulation of its so-called vertex figure.}
  \label{fig:triangulation-cone}
\end{figure}
\begin{proof}
  Since $C$ is pointed,
  there is an inequality $a^T x \leq 0$ defining its vertex $0$.\footnote{%
  That is, $a^Tx \leq 0$ is valid for $C$ and $a^Tx < 0$ for all $x \in C \setminus \{ 0 \}$.}
  Let $P := C \cap \{ a^Tx = -1 \}$
  and let $\Delta_1, \ldots, \Delta_n$ be its triangulation by Theorem~\ref{thm:polytope-triangulation},
  see Figure~\ref{fig:triangulation-cone}.
  Then the $C_i := \cone \Delta_i$ satisfy the desired properties.
\end{proof}




\section{The algebra of polyhedra}

Given a triangulation of a cone $C$ into simplicial cones $C_1, \ldots, C_n$,
it seems natural to express the generating function $f(C;x)$ as
the sum of the generating functions $f(C_i;x)$.
Note, however, that integer points on lower-dimensional cones that arise from intersections $C_i \cap C_j$
will be counted multiple cones.
The generating functions corresponding to those lower dimensional cones must then be subtracted,
but then points that have been subtracted too often must be added again, and so on.
In this section, we will put such computations on a rigorous foundation.

\begin{definition}
  We denote
  \begin{enumerate}[(a)]
    \item by $\cP^d$ the set of rational polyhedra in $\R^d$,
    \item by $\cP_0^d \subset \cP^d$ the set of rational polyhedra that contain lines,
    \item by $\cP_v^d \subset \cP^d$ the set of rational polyhedra that do not contain lines (i.e., polyhedra with vertices aka pointed polyhedra),
    \item by $\cP_b^d \subset \cP^d$ the set of bounded rational polyhedra, and
    \item by $\cP_\ell^d \subset \cP^d$ the set of rational polyhedra that is not full-dimensional.
  \end{enumerate}
\end{definition}
The relationships between those families of polyhedra are as follows:
\begin{center}
  \begin{tikzpicture}
    \draw[thick] (0,0) rectangle (6,2);
    \draw[thick] (2,0) rectangle (2,2);
    \draw (1,1) node {$\cP_0^d$};
    \draw (5,0.5) node {$\cP_v^d$};
    \draw[thick] (3.5,1) ellipse[x radius=1cm,y radius=0.5cm] node {$\cP_b^d$};
    \draw[dashed] (1,2) .. controls (2,0) .. (4,2);
    \draw (2,1.5) node {$\cP_\ell^d$};
  \end{tikzpicture}
\end{center}

\begin{definition}
  Let $P \in \cP^d$.
  The \emph{indicator} or \emph{characteristic} function of $P$ is denoted by $[P] = \chi_P : \R^d \to \R$,
  where
  \[
    [P](x) := \begin{cases}
                1 & x \in P \\
                0 & x \not\in P
              \end{cases}
  \]
  Then $\R \cP^d$ is the sub-vector space of the space of all functions
  that is generated by $\{ [P] : P \in \cP^d \}$.
  The vector spaces $\R \cP_0^d$ etc. are defined analogously.
\end{definition}

\begin{lemma}[Inclusion-exclusion formula]
  \label{lemma:inclusion-exclusion}
  Let $P_1, \ldots, P_m \in \cP^d$.
  Then
  $[A_1 \cup \dots \cup A_m] = \sum_{\emptyset \neq I \subseteq [m]} (-1)^{|I| - 1} [\bigcap_{i \in I} A_i]$.
\end{lemma}
\begin{proof}
  Left as an exercise.
\end{proof}

\begin{lemma}
  \label{lemma:generating-functions-linear-as-series}
  Let $R := \R[[x_1,\ldots,x_d,x_1^{-1},\ldots,x_d^{-d}]]$ be the vector space of formal power series in $d$ variables.\footnote{%
  This is \emph{not} a ring, because multiplication of formal power series with unbounded positive \emph{and} negative exponents is not defined.}
  There exists a unique linear map $F: \R \cP^d \to R$
  such that $F([P]) = f(P;x)$ for all $P \in \cP^d$.
\end{lemma}
\begin{proof}
  If we can show that one such $F$ exists,
  then uniqueness follows immediately because its values are fixed on a generating set of $\R\cP^d$.
  If the $[P]$ for $P \in \cP^d$ were linearly independent,
  then existence of $F$ would be obvious.
  Since they are not, we need to show that our desired values of $F$ are consistent under linear dependencies.

  To be precise, we need to show that for all linear dependencies of the form
  \[
    \sum_{i=1}^n \alpha_i [P_i] = 0
  \]
  one has
  \[
    \sum_{i=1}^n \alpha_i f(P_i; x) = 0,
  \]
  where the equality is in the sense of formal power series.

  The equality of formal power series is defined as the equality of all coefficients.
  That is,
  we need to determine the coefficient $a_p$ of the monomial $x^p$
  for every integer point $p \in \Z^d$.
  In fact,
  \[
    a_p = \sum_{i: p \in P_i} \alpha_i = \left(\sum_{i=1}^n \alpha_i [P_i] \right)(p) = 0 \text{ for all } p \in \Z^d,
  \]
  which completes the proof.
\end{proof}

\begin{corollary}
  \label{corollary:cone-triangulation-series}
  Let $C \in \cP^d$ be a pointed cone with triangulation $C_1, \ldots, C_n \in \cP^d$.
  Then
  \[
    f(C;x) = \sum_{\emptyset \neq I \subseteq [n]} (-1)^{|I| - 1} f(\bigcap_{i \in I} C_i; x)
  \]
  where the equality is understood as equality of formal power series.
\end{corollary}
\begin{proof}
  Follows from Lemmas~\ref{lemma:inclusion-exclusion} and~\ref{lemma:generating-functions-linear-as-series}.
\end{proof}

The sum in Corollary~\ref{corollary:cone-triangulation-series} typically contains
many index sets $I$ for which the intersection $\bigcap_{i \in I} C_i$ is trivial,
i.e., equal to $\{ 0 \}$.
Using tools that we will develop later, it turns out that those summands can be ignored.
Showing this is left as an exercise.

While Lemma~\ref{lemma:generating-functions-linear-as-series}
rigorously justifies computations with power series $f(P;x)$,
we really want an anologous statement for rational functions.
This is not immediate because, as we have seen in Section~\ref{sec:simple-example-generating-functions},
the series $f(P;x)$ converges nowhere for $P \in \cP_0^d$,
and so it is not clear what rational function to assign to such polyhedra.
However, it turns out that assigning rational functions to \emph{pointed} polyhedra is sufficient:

\begin{lemma}
  \label{lemma:algebra-generated-by-pointed}
  $\R \cP^d = \R\cP_v^d$.
\end{lemma}
\begin{proof}
  The inclusion from right to left is immediate.
  We need to show that $[P] \in \R\cP_v^d$ for all $P \in \cP_0^d$;
  the inclusion from left to right then follows by linearity.

  Let $Q_1, \ldots, Q_n \in \cP_v^d$ and $\varepsilon_1, \ldots, \varepsilon_n \in \R$ such that
  \[
    [\R^d] = \sum_{i=1}^n \varepsilon_i [Q_i]
  \]
  One can obtain such an equation by applying the inclusion-exclusion formula to the orthants of $\R^d$.
  We then observe that pointwise multiplication of indicator functions of polyhedra
  corresponds to taking intersections, and compute:
  \[
    [P] = [\R^d] \cdot [P] = \sum_{i=1}^n \varepsilon_i [Q_i] \cdot [P] = \sum_{i=1}^n \varepsilon_i [Q_i \cap P] \in \R\cP_v^d,
  \]
  since the subset $Q_i \cap P$ of the pointed polyhedron $Q_i$ is pointed as well.
\end{proof}
\begin{figure}
  \begin{center}
    \begin{tikzpicture}
      \begin{scope}
        \clip (-1,-1) rectangle (1,1);

        \draw[thick,fill=black!10] (-1,1.5) -- (1.5,-1) -- (1.5,1.5) -- cycle;
        \draw (-1,0) -- (1,0);
        \draw (0,-1) -- (0,1);
      \end{scope}

      \draw (1.25,0) node {$=$};

      \begin{scope}[xshift=2.5cm]
        \clip (-1,-1) rectangle (1,1);

        \draw[thick,fill=black!10] (-1,1.5) -- (0,0.5) -- (0,1.5) -- cycle;
        \draw (-1,0) -- (1,0);
        \draw (0,-1) -- (0,1);
      \end{scope}

      \draw (3.75,0) node {$+$};

      \begin{scope}[xshift=5cm]
        \clip (-1,-1) rectangle (1,1);

        \draw[thick,fill=black!10] (0,1.5) -- (0,0.5) -- (0.5,0) -- (1.5,0) -- (1.5,1.5) -- cycle;
        \draw (-1,0) -- (1,0);
        \draw (0,-1) -- (0,1);
      \end{scope}

      \draw (6.25,0) node {$+$};

      \begin{scope}[xshift=7.5cm]
        \clip (-1,-1) rectangle (1,1);

        \draw[thick,fill=black!10] (1.5,-1) -- (0.5,0) -- (1.5,0) -- cycle;
        \draw (-1,0) -- (1,0);
        \draw (0,-1) -- (0,1);
      \end{scope}

      \draw (8.75,0) node {$-$};

      \begin{scope}[xshift=10cm]
        \clip (-1,-1) rectangle (1,1);

        \draw[thick,fill=black!10] (0,0.5) -- (0,1) -- cycle;
        \draw[thick,fill=black!10] (0.5,0) -- (1,0) -- cycle;
        \draw (-1,0) -- (1,0);
        \draw (0,-1) -- (0,1);
      \end{scope}
    \end{tikzpicture}%
  \end{center}
  \caption{An illustration of the proof of Lemma~\ref{lemma:algebra-generated-by-pointed}.}
  \label{fig:algebra-generated-by-pointed}
\end{figure}




\section{Rational generating functions of polyhedra}





\section*{Exercises}

\begin{enumerate}
  \item Show Lemma~\ref{lemma:cone-polars}.

  \item Show Lemma~\ref{lemma:simplicial-cone-polars}.

  \item Show the inclusion-exclusion formula (Lemma~\ref{lemma:inclusion-exclusion}).

  \item Show the remark after Corollary~\ref{corollary:cone-triangulation-series}:
    In the expression
    \[
      f(C;x) = \sum_{\emptyset \neq I \subseteq [n]} (-1)^{|I| - 1} f(\bigcap_{i \in I} C_i; x)
    \]
    for a triangulation of a pointed cone $C$,
    the trivial summands on the right hand side add up to $0$.

    \emph{Hint:} You can apply the Euler characteristic to the inclusion-exclusion formula applied to the vertex figure of the triangulation.
\end{enumerate}





\bibliographystyle{alpha}
\bibliography{literature}

\end{document}

